In this work we have presented many different approaches to model the Portfolio Selection problem. In particular, we chose to focus the most on a risk based diversification strategy named Risk Parity. There are many formulations for Risk Parity problem, both convex or non-convex.\\ After a quick overview on the pre-existing solving algorithms for this kind of problem, we proposed a globally convergent decomposition framework that aims to find a Risk Parity solution starting from a non-convex formulation.\\ Our framework performs decomposition at two levels: a two-block Gauss-Seidel decomposition with respect to the $x$ and $\theta$ variables and then a further decomposition with respect to $x$. We proposed two slightly different versions of the decomposition algorithm that use a different approach to perform the decomposition with respect to $x$: the first uses Quadratic Line Search while the second uses an Exact Line Search on a Proximal Point modification of the problem.\\ We were able to proof the convergence properties of both the proposed methods and then we have compared them against a commercial solver in terms of computational times and quality of the Risk Parity solution produced, using different settings.\\From the experiments, it turns out that the decomposition method proposed performs particularly well compared to the commercial solver, when applied to the Least-Square problem with a particular type of box constraint ($l=0, u=1$), while it is outperformed in terms of computational time by the commercial solver when we set differents box constraints.\\ Anyway, the value of the objective function (i.e. the average deviation from Risk Parity) calculated on the critical point found by the decomposition method is generally lower than the one produced by the commercial solver. 
