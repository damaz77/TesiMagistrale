Since 1952, when Harry Markowitz wrote “Portfolio Selection", published on the Journal of Finance, many different approaches that aim to find the optimal portfolio have been proposed. What differentiates between these models is what is minimized or maximized: in some cases, models aims to maximize the expected return of the invested portfolio, in other cases they try to minimize the risk of losing the money invested; there are also models in which the objective function takes into account both the expected return and the risk of the investment. Generally, the most important drawback of these approaches is that they don't produce well diversified portfolios. Diversification means reducing risk of losing the money by investing in a variety of assets. In this thesis, after examining the pros and cons of some of the portfolio selection models in the literature, we focus on a diversification strategy called Risk Parity. The thesis introduces a globally convergent decomposition framework that can be applied to a non-convex formulation of the Risk Parity problem.