\documentclass[a4paper,12pt,oneside]{book}
\usepackage[utf8]{inputenc}
\usepackage{graphicx}
\usepackage{amsfonts}
\usepackage{amsmath}
\usepackage{amsthm}
\usepackage{dsfont}
\usepackage{multirow}
\usepackage{url}
\usepackage[ruled,vlined, linesnumbered]{algorithm2e}
\usepackage{color}
\usepackage[dvipsnames]{xcolor}
\newtheorem{definition}{Definition}[section]
\newtheorem{theorem}{Theorem}[section]
\newtheorem{lemma}[theorem]{Lemma}
\newtheorem{example}{Example}[section]
\numberwithin{equation}{section}
\usepackage{changepage}
\newcommand{\R}{\mathbb{R}}
\newcommand{\N}{\mathbb{N}}
\newtheorem{oss}{Remark}
\newtheorem{proposition}[theorem]{Proposition}
\newtheorem{corollary}[theorem]{Corollary}
\setlength\parindent{0pt}
\usepackage{pgfplots}
\usepackage{tikz}
\usepackage{caption}
\usepackage{subfig}
\tikzset{
  font={\fontsize{9pt}{12}\selectfont}}
\SetKwRepeat{Do}{do}{while}
\renewcommand{\arraystretch}{2}
\title{Optimization Methods Applied to Portfolio Selection}
\author{Federico D'Amato}
 
\begin{document}
 
\maketitle\clearpage
\tableofcontents\clearpage

\chapter{Introduction}
Portfolio selection (or portfolio management) is the art and science of making decisions about investment mix and policy, matching investments to objectives, asset allocation for individuals and institutions, and balancing risk against performance. Modern Portfolio Theory (MPT) approaches investing by examining the entire market and the whole economy. The theory is an alternative to the older method of analyzing each investment’s individual merits. When investors look at each investment individual merit, they are analyzing an investment without worrying about the way different investments will perform in relation to each other. As a matter of fact, MPT places a large emphasis on the correlation between investments. Correlation is the amount we can expect various investments, and various asset classes, to change in value compared with each other. 

\section{Risk}
The term \textit{risk} is defined as the measurement of the likelihood that an investment will go up and down in value, how often and by how much. The theory assumes that investors prefer to minimize risk; in fact, it assumes that given the choice of two portfolios with equal returns, investors will choose the one with the least risk. If investors take on additional risk, they will expect to be compensated with additional return. Risk comes in two major categories:
\begin{itemize}
\item \textbf{Systematic risk} - the possibility that the entire market and economy will show losses, which will negatively affect nearly every investment; also called \textit{market risk};
\item \textbf{Unsystematic risk} – the possibility that an investment or a category of investments will decline in value without having a major impact upon the entire market
\end{itemize}
Diversification generally does not protect against systematic risk because a drop in the entire market and economy typically affects all investments. However, diversification is designed to decrease unsystematic risk. Since unsystematic risk is the possibility that one single thing will decline in value, having a portfolio invested in a variety of stocks, a variety of asset classes and a variety of sectors will lower the risk of losing much money when one investment type declines in value.

\section{Markovitz Theory}
Consider a portfolio with $n$ different assets where asset number $i$ will give the return $R_i$. Let $\mu_i$ and $\sigma^2_i$ be the corresponding mean and variance and assume, for any two assets $i$ and $j$ that is known their correlation coefficient $\rho_{ij}$. Let $\sigma_{i,j} = \rho_{ij}\sigma_i\sigma_j$ be the covariance between $R_i$ and $R_j$. Suppose the the relative amount of the value of the portfolio invested in asset $i$ is $x_i$. If $R$ is the return of the whole portfolio then:
\begin{equation}
\mu = E[R] = \sum\limits_{i=1}^n\mu_i x_i 
\end{equation}
\begin{equation}
\sigma^2 = Var[R] = \sum\limits_{i=1}^n\sum\limits_{j=1}^n\sigma_{i,j}x_i x_j = x^T \Sigma x
\end{equation}
where $\Sigma$ is the covariance matrix and $\Sigma_{ij} = \sigma_{i,j}x_i x_j$ \cite{markovitz}.
For different choices of $x_1, ..., x_n$ the investor will get different combinations of $\mu$ and $\sigma^2$. The set of all possible ($\sigma^2$, $\mu$) combinations is called the \textit{attainable set}. Those ($\sigma^2$, $\mu$) with minimum $\sigma^2$ for a given $\mu$ or more and maximum $\mu$ for a given $\sigma^2$ or less are called the \textit{efficient set} (or efficient frontier). Since an investor wants a high profit and a small risk, he wants to maximize $\mu$ and minimize $\sigma^2$ and should therefore choose a portfolio which gives a ($\sigma^2$, $\mu$) combination in the efficient set. In Figure 1 the attainable set is the interior of the ellipse and the efficient set is the upper left part of its boundary. \\
\begin{figure}
\centering
\includegraphics[scale=0.5]{efficient_set} 
\caption{The efficient set in the ($\sigma^2$,$\mu$) plane}
\end{figure}

\section{Sharpe Ratio (SR)}
The Sharpe Ratio is a measure for calculating risk-adjusted return \cite{sharpe}. The Sharpe Ratio is the average return earned in excess of the \textit{risk-free rate} per unit of volatility or total risk. Subtracting the \textit{risk-free rate} from the mean return, the performance associated with risk-taking activities can be isolated:
\begin{equation}\label{eq:SR}
SR(x) = \frac{\mu(x) - \mu_{RF}}{\sigma(x)}
\end{equation} 
where $\mu(x)$ is the expected return of the invested portfolio, $\mu_{RF}$ is the \textit{risk-free rate} and $\sigma(x)$ is the portfolio standard deviation ($\sqrt{x^T \Sigma x}$). The greater a portfolio's Sharpe Ratio, the better its risk-adjusted performance has been. A negative Sharpe Ratio indicates that a risk-less asset would perform better than the security being analyzed.

\section{Diversification Ratio (DR)}
The \textit{Diversification Ratio} (DR) is a measure of a portfolio diversification, defined as the ratio of the portfolio’s weighted
average volatility to its overall volatility \cite{diversification}. Formally, if $x$ is the vector of the assets' weights,
\begin{equation}\label{eq:dr}
DR(x) = \frac{\sum_{i=1}^{n}x_i\sigma_i}{\sigma(x)}
\end{equation}
It is intuitive that portfolios with “concentrated” weights and/or highly correlated holdings would be poorly diversified, and hence be characterized by relatively low DRs.

\section{Portfolio selection}
The Markowitz portfolio theory states that an investor should
choose a portfolio from the efficient set, depending on how risk averse he is. One way to handle this, is to consider the optimization problem\footnotemark[1]\footnotetext[1]{The condition $x \geq 0$ is often referred as \textit{long-only} investment strategy.}:
\begin{equation}
\begin{aligned}
&\min_x &&(\sigma^2 - A\mu) = x^T \Sigma x - A\mu^T x\\
&\text{s.t.}&&\mathds{1}^T x=1\\
&&&x \geq 0
\end{aligned}
\end{equation}

where $\mathds{1} = [1, .., 1] \in \mathbb{R}^n$ is the vector of all ones and $A$ ($0 \leq A \leq \infty$) is the so called \textit{risk aversion index}. $A = 0$ will result in the portfolio with the \textit{smallest variance}. An increasing $A$ corresponds to the investor becoming more willing to take a bigger risk to get a higher expected return; $A = \infty$ corresponds to the investor only caring about getting a large expected return no matter what the risk is \cite{markovitz}.\\
If we set $A=0$, the problem became a \textit{mean-variance
optimization} (MVO) problem. If we add the constraint that every asset yields at least a target value $R$ of expected return, we obtain a convex quadratic programming problem:
\begin{equation}\label{eq:variance}
\begin{aligned}
&\min_x &&\frac{1}{2}x^T \Sigma x\\
&\text{s.t.}
&&\mu^T x \geq R\\
&&&\mathds{1}^T x=1\\
&&&x \geq 0
\end{aligned}
\end{equation}
Note that the matrix $\Sigma$ is positive semidefinite since $x^T \Sigma x$, the variance of the portfolio, must be nonnegative for every portfolio (feasible or not) $x$.\\
An alternative way to formulate the MVO problem is the following:
\begin{equation}\label{eq:variance2}
\begin{aligned}
&\max_x &&\mu^Tx\\
&\text{s.t.}
&&x^T \Sigma x \leq \sigma^2\\
&&&\mathds{1}^T x=1\\
&&&x \geq 0
\end{aligned}
\end{equation}

While in (\ref{eq:variance}) we attempt to minimize the variance of the portfolio, with the constraint that the return is at least greater than a fixed value of return $R$, in (\ref{eq:variance2}) we try to maximize the return, constraining the variance to be less or equal than a certain fixed value $\sigma^2$ \cite{libro}.\\
The main problems of the mean-variance approach are two. The first is that there is no reason to expect that solutions to the Markowitz model will be well diversified portfolios. Diversification means reducing \textit{risk} by investing in a variety of assets. This model tends to produce portfolios with unreasonably large weights in certain asset classes \cite{libro}. The second criticism of the Markovitz model is its sensitivity to inputs \cite{tutuncu}.


\chapter{Risk Parity portfolio}
Numerous methods based on the famous Markowitz mean-variance framework have been proposed to overcome its drawbacks while maintaining its advantages. One of these are the risk based diversification strategies. Unlike the classic mean-variance approach, risk based strategies do not incorporate expected returns into the formulation. Motivations for not using expected returns in the portfolio construction include the difficulty of estimating these quantities accurately, and the well documented sensitivity of the optimal weights to small changes in expected returns. 

\section{Standard formulation}
Risk parity portfolios can be motivated by considering Euler decomposition of a portfolio risk measure into contributions from each asset in the portfolio \cite{tutuncu}.
\begin{theorem}
(Euler’s theorem) Let a continuous and differentiable function f : $\mathbb{R}^n \longrightarrow \mathbb{R}$ be a homogeneous function of degree one. Then
\begin{equation}\label{eq:euler}
f(x) = x_1\cdot\frac{\partial f}{\partial x_1} + x_2\cdot\frac{\partial f}{\partial x_2} + ... + x_n\cdot\frac{\partial f}{\partial x_n}
\end{equation}
\end{theorem}
\begin{proof}
If $f$ is homogeneous of degree one, then
\begin{equation}\label{eq:p1}
f(c x_1, .., c x_n) = c \cdot f(x_1, .., x_n)
\end{equation}
$\forall (x_1, .., x_n)$  and  $c > 0$. Differentiating each side of (\ref{eq:p1}) with respect to $c$ we obtain (\ref{eq:euler}).
\end{proof}
Then the risk contribution of the asset $i$th is
\begin{equation}\label{eq:marg}
RC_i = x_i \frac{\partial \sigma(x)}{\partial x_i}
\end{equation} 
So, the total risk of the invested portfolio can be written as
\begin{equation}\label{eq:total}
\mathcal{R}(x) = \sigma(x) = \sum_{i=1}^n RC_i
\end{equation}
Let $b=(b_1,..,b_n)$ be a vector of \textit{budgets} such as $b_i > 0$ and $\sum_{i=1}^n b_i = 1$. The simplest way to define a risk parity \cite{bruder} (or risk budgeting) portfolio is through the following system of constraints:\\
\begin{equation}\label{eq:1}
\begin{aligned}
&\begin{cases}
RC_1(x_1,..,x_n) = b_1 \mathcal{R}(x)\\
..\\
RC_i(x_1,..,x_n) = b_i \mathcal{R}(x)\\
..\\
RC_n(x_1,..,x_n) = b_n \mathcal{R}(x)
\end{cases}
\end{aligned}
\end{equation} Then, the risk parity constraint is
\begin{equation}\label{eq:newconst}
RC_i = b_i \sigma (x)
\end{equation}
Another useful formulation of (\ref{eq:newconst}) is
\begin{equation}\label{eq:newconst1}
x_i \frac{\partial_{x_i} \sigma (x)}{b_i} = x_j \frac{\partial_{x_j} \sigma (x)}{b_j} \hspace{1em} \forall i,j
\end{equation}
The risk parity portfolio \cite{roncalli} is defined by this system of equations:
\begin{equation}\label{eq:rb}
\begin{aligned}
&x_i \frac{\partial_{x_i} \sigma (x)}{b_i} = x_j \frac{\partial_{x_j} \sigma (x)}{b_j} \hspace{1em} &&\forall i,j\\
&x_i \geq 0,  &&b_i > 0 \\
&\sum_{i=1}^n x_i = 1,  &&\sum_{i=1}^n b_i = 1
\end{aligned}
\end{equation}

\subsection{About the vector of budgets}
A possible difficult may appear when one specifies that some risk budgets are equal to zero. Let $\Sigma$ be the covariance matrix specified as follows:
\begin{equation}
\Sigma_{i,j} = \rho_{i,j}\sigma_i \sigma_j
\end{equation}
where $\sigma_i > 0$ is the volatility of the asset $i$ and $\rho_{i,j}$ is the cross-correlation between the assets $i$ and $j$. It comes that:
\begin{equation}
\frac{\partial \sigma(x)}{\delta x_i} = \frac{x_i \sigma_i^2 + \sigma_i \sum_{j \neq i} x_j \rho_{i,j} \sigma_j}{\sigma(x)}
\end{equation}
Suppose that the risk budget $b_k$ is equal to zero. From (\ref{eq:marg}) and (\ref{eq:newconst}) we obtain:
\begin{equation}
x_k\left( x_k \sigma_k^2 + \sigma_k \sum_{j \neq k} x_j \rho_{k,j} \sigma_j\right) = 0
\end{equation}
We obtain two solutions. The first one is $x_{k_1} = 0$, whereas the second one verifies:
\begin{equation}
x_{k_2} = - \frac{\sum_{j \neq k}x_j \rho_{k,j} \sigma_j}{\sigma_k}
\end{equation}
The only way to have $x_{k_2} > 0$ is to have some negative correlations $\rho_{k,j}$. In this case, it implies that:
\begin{equation}
\sum_{j \neq k}x_j \rho_{k,j} \sigma_j <0
\end{equation}
\begin{example}
Consider a universe of $3$ assets with a given budget vector:
\begin{equation*}
b=(0.5,0.5, 0)
\end{equation*}
Take, for example, $\sigma_1 = 0.2$, $\sigma_2 = 0.1$, $\sigma_3 = 0.05$ and $\rho_{1,2} = 0.5$, $\rho_{1,3} = -0.25$, $\rho_{2,3} = -0.25$. The two solutions of the RP problem are:
\begin{equation*}
x^{*1} = (0.33, 0.66, 0)
\end{equation*}
\begin{equation*}
x^{*2} = (0.2, 0.2, 0.4)
\end{equation*}
\end{example}
In practice, this second solution may not satisfy the investor. When he sets one risk budget to zero, he expects that he will not have the corresponding asset in his portfolio. If we would like to impose that some risk budgets are equal to zero, we first have to reduce the universe of assets by excluding the assets corresponding to these zero risk contributions \cite{intr}. From now on, we impose the strict constraint
\begin{equation}
b_i >0
\end{equation}

\subsection{Existence and uniqueness}
\begin{theorem}
The following optimization problem
\begin{equation}\label{eq:dc1}
\begin{aligned}
& y^* = \underset{y}{\text{argmin}}
&& \sqrt{y^T \Sigma y}\\
& \text{s.t.}
&&\sum_{i=1}^n b_i ln(y_i) \geq c\\
&&&y_i \geq 0 \hspace{1em} \forall i
\end{aligned}
\end{equation}
where $b_i > 0$, $\sum_i b_i = 1$, $c$ an arbitrary constant, is equivalent to problem (\ref{eq:rb})
\end{theorem} 
\begin{proof}
The Lagrangian function of the optimization problem (\ref{eq:dc1}) is
\begin{equation}
\mathcal{L}(y;\lambda,\lambda_c) = \sqrt{y^T \Sigma y} - \lambda^Ty - \lambda_c\left(\sum_{i=1}^n b_i ln(y_i) - c\right)
\end{equation}
The solution $y^*$ verifies the following first-order condition:
\begin{equation}
\partial_{y_i}\mathcal{L}(y;\lambda,\lambda_c) = \partial_{y_i}\sigma(y) - \lambda_i - \lambda_c b_i y_i^{-1} = 0
\end{equation}
and the Kuhn-Tucker conditions:
\begin{equation}
\begin{cases}
\min(\lambda_i,y_i)=0\\
\min\big(\lambda_c,\sum_{i=1}^n b_i ln(y_i) - c\big) = 0
\end{cases}
\end{equation}
Because $ln(y_i)$ is not defined for $y_i = 0$, it comes that $y_i > 0$ and $\lambda_i = 0$.\\
The constraint $\sum_{i=1}^n b_i ln(y_i) = c$ is necessarly reached (because the solution can not be $y^* = 0$), then $\lambda_c >0$ and we have:
\begin{equation}\label{eq:this}
y_i \frac{\partial_{y_i}\sigma(y)}{b_i}= \lambda_c \hspace{1em} \forall i
\end{equation}
Equation (\ref{eq:this}) is equivalent to (\ref{eq:newconst1}). We then deduce the RP portfolio by normalizing the solution $y^*$ such that the sum of weights equals one: $x_i^* =y_i^*/\sum_{i=1}^n y_i^*$
\end{proof}
Formulation (\ref{eq:dc1}) is very interesting because it demonstrates that the RP portfolio specified by the mathematical system (\ref{eq:rb}) exists, and it is unique as long as the covariance matrix $\Sigma$ is positive-definite. Indeed, it corresponds to the minimization of a quadratic function with a convex constraint.

\section{RP alternative formulations}
\subsection{Logarithmic barrier}
\begin{theorem}
Solve the following problem that incorporates a logarithmic barrier in the objective function is equivalent to find a RP solution:
\begin{equation}\label{eq:log1}
\begin{aligned}
& \underset{x}{\text{min}}
&&\frac{1}{2} x^T \Sigma x - c \sum_{i=1}^{n} b_i ln(x_i)\\
& \text{s.t.}
&& x_i > 0 \hspace{1em} \forall i
\end{aligned}
\end{equation}
where $b_i > 0$, $\sum_i b_i = 1$ and $c$ is positive constant.
\end{theorem}
\begin{proof}
Since $\Sigma$ is positive semidefinite and the logarithm function is strictly concave, the objective function is \textbf{strictly convex}. From the first order condition, the unique solution is in corrispondence of the point where the gradient of the objective function is zero:
\begin{equation}
\Sigma x - c b_i x^{-1} = 0
\end{equation}
Hence, at optimality we have
\begin{equation}\label{eq:a1}
(\Sigma x)_i = \frac{c b_i}{x_i} \Rightarrow \frac{x_i(\Sigma x)_i}{b_i} = \frac{x_j(\Sigma x)_j}{b_j}, \quad \forall i,j
\end{equation}
It is easy to see that equation (\ref{eq:a1}) is equivalent to (\ref{eq:newconst1}). 
\end{proof}
\subsection{Least-square model}
The log-barrier approach to finding risk parity solutions in the long-only setting does not immediately extend to scenarios with additional constraints or preferences. In particular, when general bounds are added, risk parity solution may not exist; moreover, the log-barrier formulation gives no guidance on how to produce feasible solution which may be “close to risk parity”. In addition, it is not clear how to extend this approach to the cases when risk parity is desirable not for individual assets but for groups of assets. The least-squares formulation for solving the risk parity problem is the following:
\begin{equation}\label{eq:bp}
\begin{aligned}
& \min_x
&& \sum_{i=i}^n \left( \frac{x_i(\Sigma x)_i}{b_i} - \theta \right)^2\\
& \text{s.t.}
&&\sum_{i=1}^n x_i =1\\
&&&x_i \geq 0
\end{aligned}
\end{equation}
Note that the formulation (\ref{eq:bp}) is \textbf{non-convex}.  If the optimization problem above has an optimal value of zero, then the RP portfolio is achieved. Otherwise, the value of $\sum_{i=i}^n \left( \frac{x_i(\Sigma x)_i}{b_i} - \theta\right)^2$  can be regarded as a minimum variance measure towards our goal. Note that, since (\ref{eq:bp}) is a non-convex problem, in theory it is hard to solve and may produce local solutions. Anyway, we have the following lemma (for the proof see \cite{tutuncu}):
\begin{lemma}\label{lem:rpsolution}
Let $F(x,\theta) = \sum_{i=i}^n \left( \frac{x_i(\Sigma x)_i}{b_i} - \theta \right)^2$. A solution pair $\{x,\theta\}$ is a global optimum with $F(x,\theta)=0$ if and only if $\nabla_xF(x,\theta) = 0$ and $\frac{\partial F}{\partial \theta} = 0$.
\end{lemma}
Lemma (\ref{lem:rpsolution}) implies that if constraints of (\ref{eq:bp}) are not considered, then first order optimality conditions determine the global optimal solution. On the other hand, when constraints are imposed, local optima and local stationary points can occur.
\section{Equally Risk Contribution portfolio (ERC)}
Simply put, an ERC portfolio is a portfolio where the total contribution of each asset to the total portfolio risk is equal \cite{tutuncu}. The ERC problem aims to find any portfolio that satisfies
\begin{equation}\label{eq:ercconst}
x_i\cdot\frac{\partial\sigma (x)}{\partial x_i} = x_j\cdot\frac{\partial\sigma (x)}{\partial x_j} = \lambda \quad \forall i,j
\end{equation}
where $\lambda$ is an unknown value. It is easy to notice that ERC is a special case of RP portfolio, because (\ref{eq:ercconst}) is equivalent to (\ref{eq:newconst1}) under the assumption that $b_i = 1/n \hspace{0.5em} \forall i$.\\
The solution of the following system of constraints
\begin{equation}\label{eq:b}
\begin{aligned}
&x_i\cdot\frac{\partial\sigma (x)}{\partial x_i} = x_j\cdot\frac{\partial\sigma (x)}{\partial x_j} \quad \forall i,j\\
&\sum_{i=1}^n x_i =1\\
&x_i \geq 0
\end{aligned}
\end{equation}
is a (normalized) ERC solution. There are also situations where ERC solutions may not exist because of the presence of additional restrictions on the portfolio weights. In such cases, the problem becomes to find portfolios that are close to equally risk contribution and for this purpose it will be important to quantify the deviation from equal contribution.

\subsection{Least-square model}
This section presents a least-squares formulation for solving the ERC problem.
\begin{equation}
\begin{aligned}
& \min_x
&& \sum_{i=i}^n \sum_{j=1}^{n}\left(x_i(\Sigma x)_i - x_j(\Sigma x)_j\right)^2\\
& \text{s.t.}
&&\sum_{i=1}^n x_i =1\\
&&&x_i \geq 0
\end{aligned}
\end{equation}
Alternatively, one can consider using penalty terms for deviations of risk contributions from their average value:
\begin{equation}\label{eq:ls1} 
\min_x \hspace{1em}\sum_{i=1}^{n}\left(x_i(\Sigma x)_i - \frac{\sum_{j=1}^{n} x_j(\Sigma x)_j}{n}\right)^2
\end{equation}
In this way, the objective function contains only $n$ elements in the sum. If we replace the average risk contribution term with a free variable $\theta$, which is also optimized, we obtain:
\begin{equation}\label{eq:ls}
\begin{aligned}
& \min_{x,\theta}
&& F(x,\theta) = \sum_{i=1}^n (x_i(\Sigma x)_i - \theta)^2\\
& \text{s.t.}
&&\sum_{i=1}^n x_i =1\\
&&&x_i \geq 0
\end{aligned}
\end{equation}
If the optimization problem above has an optimal value of zero, then the ERC portfolio is achieved. Otherwise, the value of $F(x,\theta)$ can be regarded as a minimum variance measure towards our goal. The auxiliary variable $\theta$ can always be set to its optimal value using the following lemma, however, allowing $\theta$ to be a free variable significantly simplifies the formulation \cite{tutuncu}. 
\begin{lemma} Given a solution x, there exist one and only one $\theta^*$ such that $F(x,\theta)$ is minimized, and
\begin{equation}
\theta^* = \frac{\sum_{i=1}^n x_i(\Sigma x)_i}{n}
\end{equation}
\end{lemma}

\subsection{Diversification constraint}
As in the RP case (see Sect. 3.3), we can give another formulation to find a ERC solution
\begin{equation}\label{eq:dc}
\begin{aligned}
& y^* = \underset{y}{\text{argmin}}
&& \sqrt{y^T \Sigma y}\\
& \text{s.t.}
&&\sum_{i=1}^n ln(y_i) \geq c\\
&&&y_i > 0 \hspace{1em} \forall i
\end{aligned}
\end{equation}
with $c$ an arbitrary constant. Following the same steps used in Sect. 3.3, we derive 
\begin{equation}
y_i \frac{\partial\sigma(y)}{\partial y_i} = \lambda_c
\end{equation}
We verify that risk contributions are the same for all assets \cite{erc}. We then deduce the ERC portfolio by normalizing the solution $y^*$ such that the sum of weights equals one: $x_i^* =y_i^*/\sum_{i=1}^n y_i^*$

\subsection{Logarithmic barrier}
As in the RP case (see Sect. 3.4.1), we can give another formulation to find a ERC solution
\begin{equation}
\begin{aligned}
& \underset{x}{\text{min}}
&&\frac{1}{2} x^T \Sigma x - c \sum_{i=1}^{n} ln(x_i)\\
& \text{s.t.}
&& x_i > 0 \hspace{1em} \forall i
\end{aligned}
\end{equation}
where $c$ is an arbitrary positive constant. Following the same steps used in Sect. 3.4.1, we derive

\begin{equation}\label{eq:a}
(\Sigma x)_i = \frac{c}{x_i} \Rightarrow x_i(\Sigma x)_i = x_j(\Sigma x)_j, \forall i,j
\end{equation}
It is easy to see that equation (\ref{eq:a}) is equivalent to (\ref{eq:ercconst}). 

\section{RP as a general case of other portfolios}
Let us introduce a quantity called $\beta_i$, that indicates the sensitivity of the asset $i$ to the systematic risk. The $\beta_i$ of the asset $i$ with respect to the portfolio $x$ is defined as
\begin{equation}
\beta_i = \frac{(\Sigma x)_i}{x^T \Sigma x}
\end{equation}
It means that the risk contribution $RC_i$ is equal to $x_i\beta_i\sigma(x)$. For RP portofolios, it follows that:
\begin{equation}
b_j x_i \beta_i = b_i x_j \beta_j
\end{equation}
We finally deduce that:
\begin{equation}\label{eq:3}
x_i = \frac{b_i \beta_i^{-1}}{\sum_{j=1}^n b_j \beta_j^{-1}}
\end{equation}
The weight allocated to the component $i$ is thus inversely proportional to its beta.\\
\subsection{Equally Weighted portfolio (EW)}
The Equally Weighted (EW) portfolio is the one where the capital is equally distributed among the assets. In terms of relative weights we have $x_i=\frac{1}{n}$. Clearly, the choice of the EW portfolio does not use any in-sample information nor involve any optimization approach \cite{colucci}. This portfolio is usually used as a benchmark to compare the performance of the portfolios constructed by other models.\\
The equally-weighted portfolio could be viewed as a risk parity portfolio when the risk budget is proportional to the beta of the asset. If we set $b_i = \beta_i/n$, using the result (\ref{eq:3}) follows that:
\begin{equation*}
x_i = \frac{1}{n}
\end{equation*}

\subsection{Most Diversified Portofolio (MDP)}
The definition of the diversification ratio (\ref{eq:dr}) leads to the contruction of maximally diversified long-only portfolios \cite{diversification}, defined as
\begin{equation}
\begin{aligned}
x^{MDP} = \mbox{argmax}_x &&&DR(x)\\
\text{s.t.}&&&\sum_{i=1}^{n} x_i = 1\\ 
&&&x_i \geq 0
\end{aligned}
\end{equation}
Consider a homogeneous investment universe of single assets where we have no reason to believe, \textit{ex ante}, that any single asset will reward risk more than another. In this universe, the \textit{ex ante} Sharpe ratios of single assets are identical, and thus each asset’s expected excess return (EER) is proportional to its volatility: risk is homogeneously rewarded. Denoting by $\mu_1,..,\mu_n$ the single assets' expected returns, we can define the EER for a single asset $i$ like
\begin{equation}
EER_i = \mu_i - \mu_{RF}
\end{equation}
where $\mu_{RF}$ is the \textit{risk-free rate}. As already stated, the single assets' EER statify
\begin{equation}
EER_i = k\sigma_i
\end{equation}
where $k$ is a positive constant. As such, for any portfolio of single assets with weights $x$ and expected return $\mu$,
\begin{equation}
EER = \sum_{i=1}^{n} x_i (EER_i) = k \sum_{i=1}^n x_i \sigma_i
\end{equation}
Using the definition of diversification ratio (\ref{eq:dr}), we obtain
\begin{equation}\label{eq:MDP1}
EER = k \sigma DR(x)
\end{equation}
Equation (\ref{eq:MDP1}) shows that portfolios’ EERs are proportional to their volatilities multiplied by their DRs. Dividing both sides of this equation by $\sigma$ demonstrates that in this homogenous universe, maximizing the DR is equivalent to maximizing the Sharpe ratio.\\
The MDP portfolio could be viewed as the RP portfolio such that the risk budgets are proportional to the product of the weight of the asset and its volatility\footnotemark[2]:
\footnotetext[2]{This solution is endogenous because it depends on the weights of the portfolio.}
\begin{equation*}
b_i \propto x_i \sigma_i
\end{equation*}
\subsection{Minimum Variance portfolio (MV)}
The following optimization problem minimizes the total variance of a fully-invested long-only portfolio:
\begin{equation}
\begin{aligned}
& \min_x
&& \frac{1}{2}x^T \Sigma x\vspace{-0.8em}\\
&\text{s.t.}
&&e^T x =1\\
&&&x \geq 0
\end{aligned}
\end{equation}
where $e = [1, .., 1] \in \mathbb{R}^n$ is the identity vector. Starting from the formulation seen above, using the first-order optimality conditions, we obtain
\begin{equation}
\Sigma x - \lambda - \gamma e = 0
\end{equation}
where $\lambda \in \mathbb{R}^n$ and $\gamma \in \mathbb{R}$ are the Lagrange multipliers corresponding to the constraints. Note that, complementary slackness conditions imply that if some $x_i$ is strictly larger than zero, then the corresponding $\lambda_i$ must be zero. We obtain
\begin{equation}\label{eq:first}
(\Sigma x)_i = \gamma \quad \forall i: x_i \neq 0
\end{equation}
From (\ref{eq:first}) we can write
\begin{equation}
\left(\frac{\Sigma x}{\sqrt{x^T \Sigma x}}\right)_i = \left(\frac{\Sigma x}{\sqrt{x^T \Sigma x}}\right)_j \quad \forall i,j: x_i,x_j \neq 0 
\end{equation}
If we take $\sigma (x) = \sqrt{x^T \Sigma x}$ as the total risk of the portfolio, $\frac{\partial\sigma (x)}{\partial x} = \frac{\Sigma x}{\sqrt{x^T \Sigma x}}$ is the vector of \textit{marginal risk} contributions for the assets in the portfolio. Finally we obtain
\begin{equation}\label{eq:erc1}
\frac{\partial\sigma}{\partial x_i} = \frac{\partial\sigma}{\partial x_j}\quad \forall i,j: x_i,x_j \geq 0
\end{equation}
The above condition implies that, as long as we invest in an asset, its marginal risk contribution should be the same as that of all other assets with positive weights in the portfolio \cite{tutuncu}. The minimum variance approach often leads to concentrated portfolios, i.e., encourages investors to concentrate on a small number of assets with lower risk profiles and to give up diversification. This behavior is often undesirable and this is exactly what equally risk optimization intends to overcome.\\
The minimum variance portfolio could be viewed as a RP portfolio when the risk budget is equal to the weight of the asset\footnotemark[2]:
\begin{equation*}
b_i = x_i
\end{equation*}

\section{Expected returns in RP portfolios}
Risk parity is generally presented as an allocation method unrelated to the Markowitz approach. Most of the time, these are opposed, because risk parity does not depend on expected returns. However, the risk parity approach has also been strongly criticized, because some investment professionals consider this aspect a weakness, with some active managers having subsequently reintroduced expected returns in an ad-hoc manner \cite{intr}. Consider the MVO unconstrained solution:
\begin{equation}\label{eq:ob1}
x^*(\gamma) = \mbox{argmin} \frac{1}{2} x^T \Sigma x - \gamma x^T (\mu - \mu_r)
\end{equation}
where $\gamma$ is the risk aversion parameter, $\mu$ is the vector of expected returns of the assets and $\mu_r$ is the return of the risk-free asset. This framework is particularly appealing because the objective function has a concrete financial interpretation in terms of utility functions, with the investor facing a trade-off between risk and performance. To obtain a better expected return, the investor must then choose a riskier portfolio.\\
Now, let $\mu(x) = x^T \mu$ be the expected return of the portfolio and $\pi(x) = \mu(x) - \mu_r$ its risk premium. The optimal solution (\ref{eq:ob1}) can also be formulated as follows:
\begin{equation}
x^*(c) =\mbox{argmin } \mathcal{R}(x) := - \pi(x) + c \cdot \sigma(x)
\end{equation}
where $c = \gamma^{-1} \sigma(x^*(\gamma))$. The Markowitz model is therefore equivalent to minimizing a risk measure that
encompasses both the performance dimension and the risk dimension \cite{intr}. Defining $\pi$ as the vector of risk premiums ($\pi_i = \mu_i - \mu_r$), follows that
\begin{equation}\label{eq:risk2}
\mathcal{R}(x) = -x^T \pi + c \cdot \sqrt{x^T \Sigma x}
\end{equation}
We can deduce that the expression of the marginal risk is:
\begin{equation}
\frac{\partial \mathcal{R}(x)}{\partial x_i} = -\pi_i + c \frac{(\Sigma x)_i}{\sqrt{x^T \Sigma x}}
\end{equation}
This means that:
\begin{equation}
RC_i = x_i \cdot \frac{\partial \mathcal{R}(x)}{\partial x_i} = -x_i\pi_i + c \frac{x_i(\Sigma x)_i}{\sqrt{x^T \Sigma x}}
\end{equation}

\subsection{Decomposing each asset's risk contribution}
Here the risk contribution has two components. The first component is the opposite of the performance contribution ($-x_i\pi_i$), while the second component corresponds to the standard risk contribution based on the volatility risk measure. We can therefore reformulate $RC_i$ as follows:
\begin{equation}
RC_i = -\pi_i(x) + c \sigma_i(x)
\end{equation}
with $\pi_i(x) = x_i\pi_i$ and $\sigma_i(x) = \frac{x_i \cdot (\Sigma x)_i}{\sigma(x)}$. The normalized risk contribution of asset $i$ can be defined as follows:
\begin{equation}
RC^*_i = \frac{RC_i}{\mathcal{R}(x)}
\end{equation}
Similarly, the normalized excess return contribution is:
\begin{equation}
PC^*_i = \frac{\pi_i(x)}{\pi(x)} = \frac{x_i \pi_i}{\sum_{j=1}^n x_j \pi_j}
\end{equation}
while the normalized volatility contribution is:
\begin{equation}
VC^*_i = \frac{\sigma_i(x)}{\sigma(x)} = \frac{x_i(\Sigma x)_i}{x^T \Sigma x}
\end{equation}

\begin{theorem}
The risk contribution of asset $i$ is the weighted average of the excess return contribution and the volatility contribution:
\begin{equation}
RC^*_i = (1-\omega)PC^*_i + \omega VC^*_i
\end{equation}
where the weight $\omega$ is:
\begin{equation}
\omega = \frac{c \sigma(x)}{- \pi(x) + c\sigma(x)}
\end{equation}
\end{theorem}
If $c=0$, $\omega$ is equal to zero. $\omega$ is then a decreasing function with respect to $c$ until the value
\begin{equation}
c^* = \frac{\pi(x)}{\sigma (x)} = \frac{\mu(x)-\mu_r}{\sigma(x)}
\end{equation}
which is the ex-ante Sharpe ratio $SR(x)$ (\ref{eq:SR}) of the portfolio. If $c > c^*$, $\omega$ is positive and $\omega \rightarrow 1$ when $c \rightarrow \infty$.\\
We conclude that
\begin{itemize}
\item The risk contribution of asset $i$ is a \textbf{return-based} contribution if $c$ is \textbf{lower} than the Sharpe ratio of the portfolio
\item The risk contribution of asset $i$ is a \textbf{volatility-based} contribution if $c$ is \textbf{higher} than the Sharpe ratio of the portfolio
\end{itemize}
The singularity around the Sharpe ratio implies that the value of $c$ must be carefully calibrated.

\subsection{Existence and uniqueness}
\begin{theorem}
If $c > SR^+$, where
\begin{equation}
SR^+ = \max \big\{\sup_{x \in [0,1]^n} SR(x), 0\big\}
\end{equation}
the RP portfolio, with risk defined in (\ref{eq:risk2}), exists and is unique. It is the solution of the following optimization problem:
\end{theorem}
\begin{equation}
\begin{aligned}
& y^* = {\text{argmin}}
&& \mathcal{R}(x)\\
& \text{s.t.}
&&\sum_{i=1}^n b_i ln(y_i) \geq \kappa\\
&&&y_i > 0 \hspace{1em} \forall i
\end{aligned}
\end{equation}
\emph{where $\kappa$ is an arbitrary constant \cite{intr}.}\\
The case $c \leq SR^+$ is not relevant from a financial point of view, because the risk measure of some portfolios may be negative. By leveraging these portfolios, the risk measure may be infinitely negative.

\section{Group risk parity}
An extension of the risk parity problem is the case of group risk parity where we seek parity of risk contributions from groups of assets instead of individual assets \cite{tutuncu}. This variation
is useful in the case when there are a large number of assets. Another reason to apply group risk parity is to avoid fully dense solutions (which are enforced by individual risk parity) when the number of assets is large. If $\mathcal{M}_1, ..., \mathcal{M}_s \subseteq \{1,..,n\}$ denote $s$ subsets of portfolio assets, then the marginal risk contribution of the $k$-th subset is
\begin{equation}
RC_{\mathcal{M}_k}(x) = \sum_{i \in \mathcal{M}_k} RC_i(x)
\end{equation} For group risk parity, we solve the following nonconvex problem:
\begin{equation}
\begin{aligned}
&\sum_{i \in \mathcal{M}_k} RC_i = b_k \mathcal{R}(x) \quad k=1,..,s\\
&\sum_{i=1}^n x_i = 1\\
& x_i \geq 0
\end{aligned}
\end{equation}
where $b=(b_1,..,b_s)$ is the vector of \textit{budgets} of the $s$ assets groups. We note that the group risk parity problem becomes a RP problem (\ref{eq:rb}) when the $\mathcal{M}_k$'s are all singletons and $\cup_k \mathcal{M}_k = \{1,..,n\}$.
\section{Generalized Risk Budgeting (GRB)}
In portfolio construction and analysis it is often preferable to group assets according to attributes such as asset class, country, sector and industry. The generalized risk budgeting (GRB) strategy is based on this very idea of managing the marginal risk contributions of subsets of assets to the total portfolio risk. Note that is used the term "subset" rather than "partition" since depending on the attributes used for the asset classification, assets may belong to more than one group.\\
The GRB problem can be formulated as follows \cite{sdp}:
\begin{equation}
\begin{aligned}
& \underset{x}{\text{max}}
&& \mu^Tx - \lambda\mathcal{R}(x)\\
& \text{s.t.}
&&\sum_{i \in \mathcal{M}_k} RC_i = \beta_k \mathcal{R}(x) \quad k=1,..,s\\
&&&\sum_{i=1}^n x_i = 1\\
&&&\sum_{i \in \mathcal{M}_k} x_i \geq 0, \quad k=1,..,s
\end{aligned}
\end{equation}
where $\mu \in \mathbb{R}^n$ is a vector of expected returns and $\lambda$ is a risk aversion parameter. Note that the constraint
\begin{equation}
\sum_{i \in \mathcal{M}_k} RC_i = \beta_k \mathcal{R}(x)
\end{equation}
implies that $\sum_{k=1}^s \beta_k =1$ when the $\mathcal{M}_k$'s form a partition. We note that the GRB problem becomes a RP problem (\ref{eq:rb}) when all assets have the same expected return, i.e. $\mu = \mu_0 \cdot e$ and the $\mathcal{M}_k$'s are all singletons.


\chapter{Algorithms for computing portfolios}
In Table \ref{tab:t} is shown an overview of the exitent algorithms for computing the GRB or RP weights; in Table 1, the acronym DC is used to refer to the log-barrier formulation (\ref{eq:log1}) whereas LS is used to refer to the least-square formulation (\ref{eq:bp}).
\begin{table}
\begin{center}
\begin{tabular}{| l | c| r |}
\hline
    \textbf{Portofolio} &\textbf{Formulation type}& \textbf{Algorithms}  \\ \hline
    \multirow{2}{4em}{GRB}& \multirow{2}{6em}{Non-Convex}  &SDP relaxation \cite{sdp}  \\ \cline{3-3}
     &  &AL-MCMC \cite{sdp}\\ \hline
    \multirow{7}{4em}{RP}& \multirow{2}{6em}{Convex (DC)}  &Newton-Nesterov \cite{spinu}\\\cline{3-3}     
    &  & CCD \cite{ccd}\\ \cline{2-3}

     &\multirow{5}{8em}{Non-Convex (LS)} & SCRIP \cite{scrip}\\ \cline{3-3}
    && Genetic Algorithm \cite{genetic} \\\cline{3-3}
    && ALM \cite{tutuncu}\\ \cline{3-3}
    && Gauss-Southwell \\ \cline{3-3}
    && SQP \cite{erc}\\ 
    \hline
\end{tabular}
\end{center}
\caption{Comparison between different algorithms for computing GRB/RP portfolios}
\label{tab:t}
\end{table}
\section{Newton's Method}
The algorithm is an application of Newton’s method to solve a system of nonlinear equations \cite{newton}. Writing the system in general form, one is interested in finding the solution of $F(y)=0$. Recall that we can write a linear approximation to this system around any point using a Taylor expansion:
\begin{equation}
F(y) \approx F(c) + J(c)(y-c)
\end{equation}
where $J(c)$ represents the Jacobian matrix of $F(y)$ evaluated in the point $c$. Now, since we are looking for a root of the system, we set $F(y) = 0$ and solve for $y$:
\begin{equation}\label{eq:newton}
y = c - [J(c)]^{-1}F(c)
\end{equation}
Of course this solution is only an approximation, but the idea behind the method is that repeated iterations of Equation (\ref{eq:newton}) will get us closer and closer to the optimal solution. In other words, given an approximate solution $y^{(n)}$, one can calculate
\begin{equation}
y^{(n+1)} = y^{(n)} - [J(y^{(n)})]^{-1}F(y^{(n)})
\end{equation}
and, if the method converges, then $y^{(n)}\rightarrow y$.\\
The remaining step is simply to write the ERC problem as a system of nonlinear equations. From (\ref{eq:ercconst}) and (\ref{eq:total}) we can write
\begin{equation}
\sigma(x) = \sqrt{x^T \Sigma x} = n\lambda \hspace{1em} \Rightarrow \hspace{1em} \Sigma x = (n^2\lambda^2)x^{-1}
\end{equation}
If we set $k=n^2\lambda^2$, we obtain
\begin{equation}
\Sigma x = k x^{-1}
\end{equation}
Now we can write
\begin{equation}
F(y) = F(x, k)= \begin{bmatrix}
    \Sigma x - kx^{-1}\\
    \sum_{i=1}^n x_i - 1
\end{bmatrix} = 0
\end{equation}
This represents a system with $n+1$ equation and $n+1$ variables, and its Jacobian is a $((n+1)\times(n+1))$ matrix
\begin{equation}
J(y) = J(x,k) = \begin{bmatrix}
    \Sigma + k\cdot diag(x^{-2}) & -x^{-1}\\
    e^T & 0
\end{bmatrix}
\end{equation}
where $diag(x^{-2})$ represents a diagonal matrix with elements equal to $1/x^2$. Algorithm \ref{alg:newton} illustrates the iterative process just described.

\begin{algorithm}
Start with an initial guess $x^{(0)}$, $k^{(0)}$, choose a threshold $\epsilon$\\

Define $y^{(0)} = [x^{(0)T}, k^{(0)}]^T$\\

\For{$n=0,1,..$}{

Calculate $F(y^{(n)})$, $J(y^{(n)})$ and $y^{(n+1)}$ using the formulas above\\

\eIf {$(\parallel y^{(n+1)} - y^{(n)} \parallel < \epsilon)$}{STOP}{

Go to Step 3.
}
}
\caption{Newton's algorithm}
\label{alg:newton}
\end{algorithm}


\section{Newton-Nesterov algorithm (NN)}
This Newtow based algorithm uses the correlation matrix $C$ rather than the covariance matrix $\Sigma$. Let $D$ be the diagonal matrix of $\Sigma$, then we can obtain $C$ from $\Sigma$ applying
\begin{equation}
C = D^{-1/2}\Sigma D^{-1/2}
\end{equation}
The correlation matrix C is positive definite, with
\begin{equation}\label{eq:corr}
C_{i,i} = 1, \quad |C_{i,j}| \leq 1 
\end{equation}
This method uses as objective function a slightly different version of the one of the problem (\ref{eq:log1})
\begin{equation}
F(x) = \frac{1}{2} x^T \Sigma x - \sum_{i=1}^n b_i ln(x_i)
\end{equation}
The gradient and the Hessian of F are given by
\begin{equation}\label{eq:grad}
F'(x) = \Sigma x - bx^{-1}, \quad F''(x) = \Sigma + diag(bx^{-2})
\end{equation}
Since F is strictly convex on $\mathbb{R}^n$ ($F''(x)$ is positive definite), it has at most one critical point. The equation used to find the critical point derives from (\ref{eq:grad})
\begin{equation}\label{eq:sol}
\Sigma x = bx^{-1}
\end{equation}
If $x^*$ is the solution of (\ref{eq:sol}), then $y^* := D^{1/2}x^*$ is the solution of
\begin{equation}
Cy = by^{-1}
\end{equation}
Therefore, we can solve the problem by replacing $\Sigma$ with $C$ in the objective function $F$. The next observation is that if $x^*$ satisfies (\ref{eq:sol}), then $t^{1/2}x^*$ satisfies the same equation with the vector $b$ replaced by $tb$. Therefore we can rescale the vector $b$, for purposes which will be transparent later, so that
\begin{equation}\label{eq:resc}
\min_{1 \leq i \leq n} b_i = 1
\end{equation}
Now it is necessary to introduce a the concept of \textit{self-concordant} function. Let $\Omega \subset \mathbb{R}^n$ an arbitrary open set; for a convex function $f \in C^3(\Omega)$ we define the norm 
\begin{equation}
\parallel u \parallel _x = \sqrt{\langle f''(x)u,u\rangle}
\end{equation}
\begin{definition}
A convex function $f \in C^3(\Omega)$ is self-concordant if the trilinear form $f'''(x)$ is bounded in the $\parallel \cdot \parallel _x$ norm as follows:
\begin{equation}
|f'''(x)[u,u,u]| \leq 2\parallel u \parallel_x^3 \quad \forall x \in \Omega, u \in \mathbb{R}^n
\end{equation}
\end{definition}
\begin{lemma}
Under the assumptions  (\ref{eq:corr}) and (\ref{eq:resc}), the function $F(x)$ is self-concordant
\end{lemma}
Given a starting point $x^{(0)} \in \mathbb{R}^n_+$, the sequence generated by the Newton's algorithm is defined by
\begin{equation}
x^{(k+1)} = x^{(k)} - \Delta x^{(k)}
\end{equation}
with the iteration step given by
\begin{equation}
\Delta x^{(k)} = [F''(x^{(k)})]^{-1}F'(x^{(k)})
\end{equation}
Now we define a quantity $\lambda_F$ as
\begin{equation}
\lambda_F(x) := \langle \Delta x, F'(x)\rangle^{1/2}
\end{equation}
In \cite{nesterov} is derived the region of the quadratic convergence for the Newton algorithm, which is defined as follows:
\begin{equation}
\lambda_F(x) < \lambda_*
\end{equation} 
where $\lambda_* = \frac{3-\sqrt{5}}{2}$. The \textit{damped} iteration lasts as long as $\lambda_F(x) > \lambda_*$ and the iteration is
\begin{equation}
x^{(k+1)} = x^{(k)} - \frac{1}{1+\lambda_F(x^{(k)})}\Delta x^{(k)}
\end{equation}
Algorithm \ref{alg:nesterov} summarizes what discussed above.

\begin{algorithm}
Start with an initial guess $x^{(0)}$, choose a threshold $\epsilon$\\
\For{$k=0,1,..$}{
Compute $F'(x^{(k)})$, $F''(x^{(k)})$, $\Delta x^{(k)}$, $\lambda_F(x^{(k)})$ \\
\eIf {$(\lambda_F(x^{(k)}) > \lambda_*)$}{
(\textit{Damped Phase}) $x^{(k+1)} = x^{(k)} - \frac{1}{1+\lambda_F(x^{(k)})}\Delta x^{(k)}$
}
{
\eIf{$(\lambda_F(x^{(k)}) > \lambda_*)$}{
(\textit{Quadratic Phase}) $x^{(k+1)} = x^{(k)} - \Delta x^{(k)}$
}
{
STOP
}
}
}
\caption{Newton-Nesterov algorithm}
\label{alg:nesterov}
\end{algorithm}

In \cite{spinu} the initial guess $x^{(0)}$ is the scaled equally-weighted portfolio
\begin{equation}
x^{(0)} := \frac{\sqrt{\sum_i b_i}}{\sqrt{e^T C e}} \cdot e
\end{equation}
\section{Cyclical coordinate descent (CCD)}The main idea behind the cyclical coordinate descent (CCD) algorithm is to minimize a function $f(y_1,...,y_n)$ by minimizing only one direction at each step, whereas classical descent algorithms consider all the directions at the same time. In this case, we find the value of $y_i$ which minimizes the objective function by considering the values taken by $y_j$ for $j \neq i$ as fixed. The procedure repeats for each direction until the global minimum is reached. This method uses the same principles as Gauss-Seidel or Jacobi algorithms for solving linear systems  \cite{ccd}.\\
A RP solution satisfies (\ref{eq:this}), that we can rewrite in the following manner
\begin{equation}
\frac{(\Sigma y)_i}{\sigma(y)} - \frac{\lambda_c b_i}{y_i} = 0
\end{equation}
Without loss of generality, we can fix $\lambda_c = 1$ and we obtain
\begin{equation}
y_i(\Sigma y)_i - b_i\sigma(y) = 0
\end{equation}
It follows that
\begin{equation}
y_i^2\sigma_i^2 + y_i\sigma_i\sum_{i \neq j} y_j \rho_{i,j} \sigma_j - b_i\sigma(y) = 0
\end{equation}
By definition of the RP portfolio we have $x_i > 0$. We notice that the polynomial function is convex because we have $\sigma_i^2 > 0$. Since the product of the roots is negative ($-b_i\sigma_i^2\sigma(y)<0$), we always have two solutions with opposite signs. We deduce that the solution is the positive root of the second degree equation:
\begin{equation}\label{eq:iter}
y_i^* = \frac{-\sigma_i\sum_{i \neq j} y_j \rho_{i,j} \sigma_j + \sqrt{\sigma_i^2(\sum_{i \neq j} y_j \rho_{i,j} \sigma_j)^2 + 4b_i\sigma_i^2\sigma(y)}}{2\sigma_i^2}
\end{equation}
If the values of $(y_1,.., y_n)$ are strictly positive, it follows that $y_i^*$ is strictly positive. The positivity of the solution is then achieved after each iteration if the starting values are positive. The coordinate-wise descent algorithm consists in iterating the equation (\ref{eq:iter}) until convergence. If we rewrite equation (\ref{eq:iter}) as follows:
\begin{equation}
y_i^* = \frac{-(\Sigma y)_i + y_i\sigma_i^2 + \sqrt{((\Sigma y)_i - y_i \sigma_i^2)^2 + 4\sigma_i^2b_i\sigma(y)}}{2\sigma_i^2}
\end{equation}
we deduce that $\Sigma y$ and $\sigma (y) $ must be computed at each iteration of the algorithm. We note $y = (y_1,.., y_{i-1}, y_i, y_{i+1},..y_n)$ and $\tilde{y} =(y_1,.., y_{i-1}, y_i^*, y_{i+1},..y_n)$ the vector of weights before and after the update of the $i$th weight $y_i$. Simple algebra shows that
\begin{equation}
\Sigma \tilde{y} = \Sigma y - \Sigma_{.,i} y_i + \Sigma_{.,i}\tilde{y}_i
\end{equation}
and
\begin{equation}
\sigma(y) = \sqrt{\sigma^2(y) - 2y_i\Sigma_{i,.}y + y_i^2\sigma_i^2 + 2\tilde{y}_i\Sigma_{i,.}\tilde{y} -  \tilde{y}_i^2\sigma_i^2}
\end{equation}
where $\Sigma_{i,.}$ and $\Sigma_{.,i}$ are the $i$th row and column of $\Sigma$. Updating $\Sigma y$ and $\sigma(y)$ is then
straightforward and reduces to the computation of two vector products. These operations dramatically reduce the computational time of the algorithm.

\section{Successive Convex Optimization method (SCRIP)}
Consider a general formulation for the RP problem
\begin{equation}\label{eq:gen}
\begin{aligned}
& \underset{x}{\text{min}}
&& \mathcal{R}(x)\\
& \text{s.t.}
&&\mathds{1}^T x = 1\\
&&&x_i \geq 0
\end{aligned}
\end{equation}
where $\mathcal{R}(x)$ measures the risk concentration and has the form
\begin{equation}
\mathcal{R}(x) := \sum_{i=1}^n (g_i(x))^2
\end{equation}
in which each $g_i(x)$ is a smooth differentiable nonconvex
function that measures the risk concentration of the $i$th asset \cite{scrip}. For instance, if we set
\begin{equation}
g_i(x) = x^T (\Sigma x)_i - \theta
\end{equation}
problem (\ref{eq:gen}) becomes (\ref{eq:ls}). Introducing a function $P(x;x^{(k)})$, that is an approximation of $\mathcal{R}(x)$, defined as
\begin{equation}
P(x;x^{(k)}) := \sum_{i=1}^n (g_i(x^{(k)}) + (\nabla g_i(x^{(k)}))^T (x-x^{(k)}))^2
\end{equation}
at the $k$th iteration, the SCRIP method aims to solve 
\begin{equation}\label{eq:scrip1}
\begin{aligned}
& \underset{x}{\text{min}}
&&P(x;x^{(k)}) + \frac{\tau}{2}\parallel x - x^{(k)} \parallel^2_2\\
& \text{s.t.}
&&\mathds{1}^T x = 1\\
&&&x_i \geq 0
\end{aligned}
\end{equation}
where $\tau > 0$ is the parameter for the regularization term. The beauty of the approximation $P(x;x^{(k)})$ is that it is an
easily computable quadratic convex function and has the same
gradient as $\mathcal{R}(x)$ at each iteration point $x^{(k)}$:
\begin{equation}
\nabla_x P(x;x^{(k)})|_{x=x^{(k)}} = \nabla_x \mathcal{R}(x)|_{x=x^{(k)}}
\end{equation}
Algorithm \ref{alg:scrip} summarizes the sequential solving approach based on a successive convex optimization method.

\begin{algorithm}
Start with an initial guess $x^{(0)}$, choose $\tau > 0$, $\{\gamma^k\}>0$\\
\Do{not convergence}{
Solve (\ref{eq:scrip1}) to get the optimal solution $\hat{x}^{(k)}$\\
Compute $x^{(k+1)} = x^{(k)} + \gamma^k(\hat{x}^{(k)} - x^{(k)}$)\\
}
\caption{SCRIP algorithm}
\label{alg:scrip}
\end{algorithm}


\section{Genetic algorithm}
In this section is presented a standard genetic algorithm to compute ERC optimal portfolios. The \textit{fitness} definition in the ERC setting is given by the deviance of each risk contribution from the mean of all risk contributions\footnotemark[3]. We use the shorthand notation of $\Delta_i = \partial_{x_i} \sigma (x)$, so we can compute the expectation $\Delta = E(\Delta_i)$ and define the fitness $f$ as the sum of the quadratic distance of each risk contribution from the mean, as in ({\ref{eq:ls1}) \cite{genetic}. This nonnegative fitness value $f$ has to be minimized, where
\begin{equation}
f= \sum_i (\Delta_i - \Delta)^2
\end{equation}
\footnotetext[3]{It is simple to extend this approach to the more general RP case}
\hspace{-0.5em}We use chromosomes of length $n$ which contain the specific portfolio weights of the $n$ risky assets. Thus, an important operator is the \textit{repair} operator, i.e. the sum of the portfolio is normalized to $1$ after each operation. The genetic operators used in the algorithm can be summarized as follows:\\[1\baselineskip]
\textbf{Selection: }The best $n_S$ chromosomes of each population are kept in the population.\\[1\baselineskip]
\textbf{Mutation: }A random selection of $n_M$ chromosomes of the parent population will be mutated. Up to a number of 15\% of the length of the respective chromosome will be changed to a random
value between the portfolio bounds. The mutation positions will be chosen randomly. Afterwards the randomly selected positions will be replaced with a random value between the upper and the lower investment limit of the respective asset.\\[1\baselineskip]
\textbf{Random Addition: }$n_R$ new and completely random chromosomes are added to each new population.\\[1\baselineskip]
\textbf{Intermediate Crossover: }Two chromosomes from the parent population will be randomly selected for an intermediate crossover. The mixing parameter between the two chromosomes will also be chosen randomly. $n_{IC}$ crossover children will be added to the next population. Let the mixing parameter be $\alpha$ and the two randomly chosen parent chromosomes $p_1$ and $p_2$ with genes $p_{1,1},...,p_{1,n}$ and $p_{2,1},...,p_{2,n}$. An intermediate crossover will result in a child chromosome $c$ where the genes are set to
\begin{equation}
c_i = \alpha p_{1,i} + (1-\alpha)p_{2,i} \hspace{1em} \forall i=1,..,n
\end{equation}
\textbf{Local Search: }In a second step, a local search algorithm is applied to the best solution of the genetic algorithm. Thereby, within each iteration of the algorithm each asset weight of the $n$ assets of the portfolio is increased or decreased by a factor $\epsilon$. Each of these $(2\times n)$ new portfolios is normalized and if one exhibits a lower fitness value then this new portfolio will be used subsequently. The algorithm terminates if no local improvement is possible anymore or the maximum number of iterations
has been reached.

\section{Minimum variance with ERC}
In this section we focus on finding the ERC solution with the least variance. Hence we consider the following problem where the objective function is a weighted sum of total variance and least-squares ERC term\footnotemark[3]:
\begin{equation}\label{eq:mvrp}
\begin{aligned}
& \underset{x,\theta}{\text{min}}
&&\sum_{i=1}^n (x_i(\Sigma x)_i - \theta)^2 + \rho x^T \Sigma x\\
& \text{s.t.}
&&a_i \leq x_i \leq b_i\vspace{0.5em}\\
&&&\mathds{1}^T x = 1
\end{aligned}
\end{equation}
Where $a_i$ and $b_i$ are nonnegative constants representing the bounds on the weight of $i$th asset and $\rho \geq 0$ is the weight parameter. In the above formulation we simply added a convex term to the objective function of (\ref{eq:ls}). In Algorithm \ref{alg:minvar} is described an approach of finding a ERC solution with the smallest variance, where the problem (\ref{eq:mvrp}) is simply solved with decreasing values of $\rho$.

\begin{algorithm}
Choose $\rho^0 > 1$, $\beta \in (0,1)$ and $x^0 $\\
\For{$k=0, 1,..$}{
$x^{k+1} :=$ argmin $F(x)$, where $F(x)$ is defined as (\ref{eq:mvrp}), using $x^k$ as starting point. \\
\eIf{($\rho^k \leq \epsilon$)}{
$x^{k+1} :=$ argmin $F(x)$ with $\rho^{k+1}=0$ using $x^k$ as starting point\\
STOP
}
{
$\rho^{k+1} := \rho^k\beta$
}
}
\caption{Sequential min-variance algorithm}
\label{alg:minvar}
\end{algorithm}


By setting initial $\rho$ to a large value, Algorithm \ref{alg:minvar} is initiated with an easy to solve problem and a solution that is close to the min-variance, but may be far from an equally risk contribution solution. Then, the algorithm solves a sequence of subproblems of the form (\ref{eq:mvrp}) with decreasing $\rho$, initializing each new subproblem with the solution of the previous subproblem.

\section{Alternating Linearization Method (ALM)}
Consider optimizing the following function
\begin{equation}\label{eq:alm}
\min_{x \in \chi, \theta} F(x,\theta) = \sum_{i} ((A_i x)^T(B_i x) -\theta)^2
\end{equation}
where $x \in \mathbb{R}^n $, $A_i, B_i \in \mathbb{R}^{m * n} $ and $\chi$ is a set defined by linear constraints. Note that all the ERC formulations seen above can be embedded into this more general formulation. The objective function of ERC problem, in the formulation of (\ref{eq:ls}), has $A_i = \Sigma_i \in \mathbb{R}^{1 * n} $ as the $i$th row of the covariance matrix, and $B_i = e_i \in \mathbb{R}^{1 * n} $ is the $i$th column of the identity. If we set $M_i = A_i^TB_i \in \mathbb{R}^{n * n}$, (\ref{eq:alm}) is equivalent to
\begin{equation}\label{eq:alm2}
\min_{x \in \chi, \theta} F(x,\theta) = \sum_{i=1}^n F_i(x) = \sum_{i=1}^n (x^TM_ix - \theta)^2
\end{equation}
Clearly $M_i$ is not generally symmetric or positive semidefinite. Hence we have a nonconvex function $F(x,\theta)$; we consider a variable splitting approach which replaces $F(x,\theta)$ by $F(x,y,\theta) = \sum_{i=1}^n (x^TM_iy - \theta)^2$, $y=x$. The method generates two sequences $\{x^k\}$ and $\{y^k\}$ in such a way that $x^k \rightarrow x^*$ and/or $y^k \rightarrow x^*$, where $x^*$ is a stationary point of (\ref{eq:alm}).\\
Given $y^k$ we have
\begin{equation}
F(x,y^k,\theta) \equiv \sum_{i=1}^n (x^TM_iy^k - \theta)^2
\end{equation}
and given $x^k$ we have
\begin{equation}
F(x^k,y,\theta) \equiv \sum_{i=1}^n ((x^k)^TM_iy - \theta)^2
\end{equation}
Both $F(x,y^k,\theta)$ and $F(x^k,y,\theta)$ are convex functions of $x$ and $y$, respectively, for any given $y^k$ and $x^k$. Let $\nabla_iF$ denote the partial derivative of $F$ with respect to $x$ ($i=1$) or $y$ ($i=2$). Using the form of (\ref{eq:alm2}) we have
\begin{equation}
\begin{split}
\nabla_1F(x,y,\theta) = \sum_{i=1}^n 2(x^TM_iy-\theta)M_iy \\
\nabla_2F(x,y,\theta) = \sum_{i=1}^n 2(x^TM_iy-\theta)M_i^Tx
\end{split}
\end{equation}
Then, the following two approximations of $F(x,y,\theta)$ are constructed:
\begin{equation}
Q^1_\mu (x,y^k) := F(x,y^k) + \langle\nabla_2F(y^k,y^k),x-y^k\rangle + \frac{1}{2\mu}\Vert x-y^k \Vert^2_2
\end{equation}
\begin{equation}
Q^2_\mu (x^{k+1},y) := F(x^{k+1},y) + \langle\nabla_1F(x^{k+1},x^{k+1}), y- x^{k+1}\rangle + \frac{1}{2\mu}\Vert x^{k+1}-y \Vert^2_2
\end{equation}
where $\mu$ is some positive scalar. The simple version of ALM algorithm is shown in Algorithm \ref{alg:alm}.

\begin{algorithm}
Choose $\mu_1^0 = \mu_2^0$, and $x^0 = y^0$\\
\For{$k=0,1,..$}{
$x^{k+1} :=$ argmin$_x \hspace{0.5em} Q^1_{\mu_1^k}(x,y^k)$\\
$y^{k+1} :=$ argmin$_y \hspace{0.5em} Q^2_{\mu_2^k}(x^{k+1},y)$\\
Choose new penalty parameters $\mu_1^{k+1} \in (0, \mu_1^k), \text{ } \mu_2^{k+1} \in (0, \mu_2^k)$
}
\caption{Alternating linearization method (ALM)}
\label{alg:alm}
\end{algorithm}

Each minimization step is a solution of a strictly convex quadratic programming problem, which can be done efficiently by many methods.\\ 
In practice backtracking strategies should be applied to choose values of parameter $\mu$ at each iteration. A practical backtracking scheme is shown in Algorithm \ref{alg:almbktr}. Note that in each minimization step, we check whether a sufficient reduction has been obtained. If so, the minimization step is accepted and $\mu$ remains the same, otherwise it is decreased and a new candidate step is computed.

\begin{algorithm}
Choose $\mu_1^0 = \mu_2^0$, and $x^0 = y^0$\\
\For{$k=0,1,..$}{
$x^{k+1} :=$ argmin$_x \hspace{0.5em} Q^1_{\mu_1^k}(x,y^k)$\\
\eIf{$F(x^{k+1}) \leq Q^1_{\mu_1^k}(x^{k+1},y^k)$}{
$\mu_1^{k+1} = \mu_1^{k}$
}
{
Find the smallest $n$ s.t. $\overline{\mu} := \mu_1^k\beta^n$,  $\overline{x}:=$ arg $\min_x Q^1_{\overline{\mu}}(x,y^k)$ and $F(\overline{x}) \leq Q^1_{\overline{\mu}}(\overline{x},y^k)$\\
$\mu_1^{k+1} = \mu_1^k \beta^n$\\
$x^{k+1} =$  arg $\min_x Q^1_{\mu_1^{k+1}}(x,y^k)$
}

$y^{k+1} :=$ argmin$_y \hspace{0.5em} Q^2_{\mu_2^k}(x^{k+1},y)$\\
\eIf{$F(y^{k+1}) \leq Q^2_{\mu_2^k}(x^{k+1},y^{k+1})$}{
$\mu_2^{k+1} = \mu_2^{k}$
}
{
Find the smallest $n$ s.t. $\overline{\mu} := \mu_2^k\beta^n$,  $\overline{y}:=$ arg $\min_y Q^2_{\overline{\mu}}(x^{k+1},y)$ and $F(\overline{y}) \leq Q^2_{\overline{\mu}}(x^{k+1},\overline{y})$\\
$\mu_2^{k+1} = \mu_2^k \beta^n$\\
$y^{k+1} =$  arg $\min_y Q^2_{\mu_2^{k+1}}(x^{k+1},y^{k+1})$
}
}
\caption{ALM with Backtracking (ALM-BKTR)}
\label{alg:almbktr}
\end{algorithm}


\section{SDP Relaxation for the GRB Problem}
In this section is explained a semidefinite programming (SDP) relaxation for the GRB problem to obtain an upper bound on the optimal objective function value. First of all, recall that is we use volatility as the risk measure, from (\ref{eq:marg}) the marginal risk contributions of the individual assets then satisfy
\begin{equation}
RC_i= x_i \frac{(\Sigma x)_i}{\sqrt{x^T \Sigma x}}
\end{equation}
We can rewrite the GRB problem in the following equivalent form:
\begin{equation}\label{eq:sdp}
\begin{aligned}
& \underset{x,X}{\text{max}}
& & \mu^Tx - \lambda(x^T\Sigma x) \\
& \text{s.t.}
&& \sum_{i \in \mathcal{M}_k} \mbox{tr}(\Gamma_i X) = \beta_k \mbox{tr}(\Sigma X), \quad k=1,..,s \\
&&&X = xx^T,\\
&&&\mathds{1}^T x = 1,\\
&&&\sum_{i \in \mathcal{M}_k} x_i \geq 0, \quad k=1,..,s.
\end{aligned}
\end{equation}
where $\Gamma_i = e^T e \Sigma$ and tr($\cdot$) denotes the trace of a matrix. Since $X = x x^T$ is the only non-convex constraint in (\ref{eq:sdp}), we obtain a convex relaxation of the GRB problem by relaxing this constraint to $X \succeq  x x^T$. We then obtain the following SDP relaxation of the GRB problem:
\begin{equation}
\begin{aligned}
& \max_{x,X}
& & \mu^Tx - \lambda(x^T\Sigma x) \\
& \text{s.t.}
&& \sum_{i \in \mathcal{M}_k} \mbox{tr}(\Gamma_i X) = \beta_k \mbox{tr}(\Sigma X), \quad k=1,..,s \\
&&&\begin{bmatrix}X & x^T\\x & 0\end{bmatrix}\succeq 0,\\
&&&\mathds{1}^T x = 1,\\
&&&\sum_{i \in \mathcal{M}_k} x_i \geq 0, \quad k=1,..,s.
\end{aligned}
\end{equation}
where the Schur complement is used to reformulate the semidefinite constraint $X \succeq  x x^T$ as a linear matrix inequality.

\section{The AL-MCMC Algorithm}
\subsection{Markov chain Monte Carlo (MCMC)}
Let $\Omega$ be the state space and $p(x) = p^*(x)/C$ denote some target probability distribution on $\Omega$ where 
\begin{equation}
C := \int_{\Omega} p^*(x) dx
\end{equation}
The MCMC method is an approach to sample from $p(x)$ when the normalizing constant is hard compute. In the MCMC approach, one
constructs a Markov chain on $\Omega$ using a "proposal" distribution $q(x_{t+1}|x_t)$ in such a way that $p(x)$
is the unique stationary distribution for the Markov chain. The main requirement of MCMC is that the unnormalized distribution, $p^*(x)$, should be easy to compute. \\
Given a current sample $x_t$ at time $t$, the proposal distribution $q(\cdot | x_t)$ is used to generate
a candidate sample, $x_{t+1}$, which is then accepted with probability
\begin{equation}\label{eq:acc}
\alpha(x_t, x_{t+1}) := \min \Big\{ 1, \frac{q(x_t|x_{t+1})p^*(x_{t+1})}{q(x_{t+1}|x_t)p^*(x_t)}\Big\}
\end{equation}
If the candidate point $x_{t+1}$ is rejected we then set $x_{t+1} = x_t$ and continue sampling in this manner.\\
Let $\mathcal{F}$ be the feasible region of the GRB problem:
\begin{equation}
\mathcal{F} := \{ x \in \mathcal{X}\quad|\quad h(x) = 0\}
\end{equation}
where $\mathcal{X} = \{x \in \mathbb{R}^n | x_i \geq 0$ $\forall i=1,..,n$, $e^T\cdot x = 1\}$. One possibility would be to set
\begin{equation}
p^*(x) = e^{\gamma \mathcal{R}(x)}\mathds{1}_{\mathcal{F}(x)} 
\end{equation}
where $\mathds{1}_{\{\cdot\}}$ is the characteristic function of a set. Since the feasible region $\mathcal{F}$ of the GRB problem is typically very "small", $p^*(x_{t+1})$ is likely to be zero for most candidate points $x_{t+1}$ and these points will be rejected in the acceptance-rejection step (\ref{eq:acc}). Therefore, using MCMC to sample only from the feasible region is very difficult, and particularly so for high-dimensional problems. One possible approach to overcoming these difficulties is to allow the MCMC iterates $x_t$ to be infeasible; by adding a term which penalizes infeasibility to our definition of $p^*$, we can "direct" them towards the feasible region. In particular, we could define
\begin{equation}
P_c(x) := \mathcal{R}(x) - \frac{1}{2} c \parallel h(x) \parallel^2_2
\end{equation}
where $c$ is a positive constant. Now we can use
\begin{equation}
p^*(x) = e^{\gamma P_c(x)}
\end{equation}
as the unnormalized density. The main difficulty with the penalty approach is that it is very sensitive to the value of the penalty parameter $c$.

\subsection{Augmented Lagrangian MCMC}
The augmented Lagrangian function of the GRB problem is defined as:
\begin{equation}
\mathcal{L}_{c_t}(u_t, x) := \mathcal{R}(x) + u_t^T h(x) + \frac{1}{2} c_t \parallel h(x) \parallel^2_2
\end{equation}
where $u_t = (u_{t,1},..,u_{t,s}) \in \mathbb{R}^s$  is a vector of time $t$ Lagrange multipliers. We define the time $t$ target distribution to be
\begin{equation}
p^*(x) = e^{\gamma \mathcal{L}_{c_t}(u_t, x)}
\end{equation}

\subsection{Description of the algorithm}
The initial vector of Lagrange dual multipliers $u_0$ and the penalty parameter $c_0$ are specified \textit{ex ante}. The values for dual multipliers $u_t$ and the non-increasing penalty parameter $c_t$ for $t\geq 1$ are chosen adaptively during the course of the simulation. In particular, $c_t$ is decreased by
a predetermined value $\epsilon_c$ when there is no improvement in constraint violations over a particular iteration. When there is an improvement in constraint violation, $c_t$ is not updated, but instead it is updated the Lagrange multipliers $u_t$ using the first order conditions, i.e.
\begin{equation}
u_{t+1} = u_t - \epsilon_u \nabla d_{c_t}(u_t)
\end{equation}
where $d_{c_t}(u) := \max_{x \in \mathcal{X}}\mathcal{L}_{c_t}(u, x) $ denotes the dual function, and $\epsilon_u$ is a given step size. $c_t$ and $u_t$ are never updated both in the same iteration, in order to ensure that we leave the current location only after adequately exploring its neighbourhood.\\
The proposal distribution $q(x_{t+1}|x_t)$ is based on a random
walk chain. In particular, one value $z^*_t \sim N(0, \sigma^q_t I )$ is generated, where $I \in \mathbb{R}^{n*n}$ is the identity matrix, and then is taken
\begin{equation}
x_{t+1} = x_t + z_t^*
\end{equation}
as the candidate point, which is then accepted with probability $\alpha(x_t, x_{t+1})$. The basic idea is to increase $\sigma^q_t$ when the acceptance rate is too high and decrease $\sigma^q_t$ when the acceptance rate is too low.\\
In each iteration, indipendently of whether the proposed sample $x_{t+1}$ is accepted or rejected, the annealing parameter $\gamma_t$ is increased according to:
\begin{equation}
\gamma_t = \sigma_{\gamma}\gamma_{t-1}
\end{equation}
where $\sigma_{\gamma} = \big(\frac{\gamma_{max}}{\gamma_0}\big)^{\frac{1}{T}}$. Thus, the AL-MCMC algorithm is a \textit{simulated annealing} algorithm. A complete specification of the AL-MCMC algorithm is given in Algorithm \ref{alg:almcmc}.

\begin{algorithm}
Choose $x_0, \gamma_0, \sigma_{\gamma}, \epsilon_c, \epsilon_u, \delta, \sigma_0^q, c_0, u_0, \kappa$\\
\For{$n=0,1,..$}{
Generate a candidate sample $x_{t+1}$ from the proposal $q(x_{t+1}|x_t)$\\
Compute $\alpha = \alpha(x_t,x_{t+1})$\\
\eIf{$\alpha \geq 1$}{
$x_{t+1} \leftarrow x_{t+1}$
}
{
Choose $p \sim \mathbb{U}[0,1]$\\
\eIf{ $p \leq \alpha$ }{
$x_{t+1} \leftarrow x_{t+1}$
}
{
$x_{t+1} \leftarrow x_{t}$
}
}
$\gamma_{t+1}\leftarrow \sigma_{\gamma}\gamma_t$\\
\eIf{$\parallel h(x_{t+1})\parallel^2_2$ $<$ $\parallel h(x_{t})\parallel^2_2$}{
$u_{t+1}\leftarrow u_t - \epsilon_u \nabla d_{c_t}(u_t)$\\
\If{$\frac{\parallel h(x_{t})\parallel^2_2}{\parallel h(x_{t+1})\parallel^2_2} - 1 > \delta $}
{
$\sigma_{t+1}^q \leftarrow \kappa \sigma_t^q$
}
}
{
$c_{t+1} \leftarrow c_t + \epsilon_c$
}
}
\caption{AL-MCMC algorithm}
\label{alg:almcmc}
\end{algorithm}

\section{Equal Risk Bounding (ERB)}
Since the ERC approach requires all assets to give an equal risk contribution to the variance of the selected portfolio, it is clear that the ERC portfolio must contain \textit{all} assets\footnotemark[4].\footnotetext[4]{This is also true in the RP case} However, this might not be sensible if the exclusion of some asset provides a less risky portfolio. When aiming at minimizing risk, one should more rationally require that the risk contributions of all assets to the variance should not exceed a given threshold which might then be minimized. This alternative approach, which is called \textit{Equal Risk Bounding} (ERB), might, and actually does in some cases, select portfolios that do not contain all assets and where the total risk contributions of all assets is strictly smaller than in the ERC portfolio \cite{erb}. The Equal Risk Bounding approach can be formulated as follows:
\begin{equation}\label{eq:erb1}
\begin{aligned}
& \min_x
& & \lambda\\
& \text{s.t.}
& & x_i (\Sigma x)_i \leq \lambda\\
&&&\mathds{1}^T x =1\\
&&&x_i \geq 0
\end{aligned}
\end{equation}
Note that this is a nonconvex quadratic programming problem with quadratic constraints which might have \textit{multiple} solutions. Let $(x^{ERB},\lambda^{ERB})$ be a solution of the above problem. Then $\lambda^{ERB}$ is an upper bound for the total contribution of each asset to the variance. Thus the ERC portfolio is theoretically dominated by the ERB portfolios in terms of variance.\\
The solution $(x^{ERC},\lambda^{ERC})$ of the ERC problem (where $\lambda^{RP} = \frac{x^{T} \Sigma x}{n}$) is also a feasible solution to the ERB problem. Thus
\begin{equation}
\lambda^{ERB} \leq \lambda^{ERC} \Rightarrow \sigma^2(x^{ERB}) \leq \sigma^2(x^{ERC})
\end{equation}
The stricly inequality can actually occur, but only when $x_i=0$ for some $i$.\\
Problem (\ref{eq:erb1}) is an hard nonconvex quadratic programming problem with quadratic constraints. However its solutions, the Equal Risk Bounding portfolios, have a particular structure which can be exploited to reduce the problem of solving (\ref{eq:erb1}) to the problem of solving a finite number of simpler ERC problems.
\begin{lemma}
Let ($x^{ERB},\lambda^{ERB}$) be a solution of the ERB problem (\ref{eq:erb1}). If $x_i^{ERB}>0$, then
\begin{equation}
x_i^{ERB}(\Sigma x^{ERB})_i = \lambda^{ERB}
\end{equation}
\end{lemma}
This result clearly implies that for every solution to the ERB model, there exists a subset $P$ of the set $N = \{1,...,n\}$ of indices such that only the assets with indices in $P$ have nonzero weights, and such weights are those of the ERC portfolio of the assets with indices in P.
\begin{theorem} 
Every solution to the ERB problem (\ref{eq:erb1}) has the form ($x_P,x_Z$) where $P \cup Z=N$, $P \cap  Z = \emptyset$, $x_Z=0$ and $x_P$ is the unique solution to the ERC problem (\ref{eq:b}) with respect to the indices in $P$
\end{theorem}
Thanks to the Theorem 3.4, a solution to (\ref{eq:erb1}) can be obtained by solving a nonlinear pseudo-boolean optimization problem. For a subset $P\subseteq N = \{1,...,n\}$ of indices, let $x_i^{RP}(P)$ denote the asset weights and $\lambda^{ERC}(P)$ denote the equal risk contribution of the assets with indices in P of the (unique) ERC portfolio. Any ERB portfolio $x^{ERB}$ is given by
\begin{equation}
x_i^{ERB} = \begin{cases}
        x_i^{ERC}(P) \hspace{2em} \text{if}\hspace{1em} i \in P
        \\
        0 \hspace{5em} \text{otherwise}
        \end{cases}
\end{equation}
for any P which solves the pseudo-boolean optimization problem
\begin{equation}\label{eq:c}
\lambda^{ERB} = \min_{P\subseteq N} \lambda^{ERC}
\end{equation}
Clearly a brute force solution of the pseudo-boolean optimization problem (\ref{eq:c}) requires the solution of an exponential number of ERC models. However, more refined exact methods and heuristics have been developed for pseudo-boolean optimization problems. Furthermore, some experiments has found that for the ERB model the optimal solution can almost always be obtained by considering only subsets $P$ with at least $n-3$ indices. However, the theoretical complexity of the problem of exactly finding an ERB portfolio is still unknown in the general case.

\chapter{A decomposition framework}
\section{Preliminary background}
Let us consider the following optimization problem:
\begin{subequations}\label{eq:problem} 
\begin{align}
\min_{x,y} & \quad f(x,y)  \\
\text{s.t.} & \quad l \leq x \leq u \\
& \quad \mathds{1}^T x = 1 
\end{align}
\end{subequations}
where $x \in \R^n$, $y \in \R^m$, $f$ continuously differentiable, $l, u \in \R^n$ with $l < u$.\\
We define the feasible set $\mathcal{F}$  of Problem (\ref{eq:problem}):
\begin{equation}
\mathcal{F} = \{(x,y) \in \R^{n+m} : \mathds{1}^T x = 1, l \leq x \leq u\}.
\end{equation}
Since the constraints of Problem (\ref{eq:problem}) respect constraints qualification conditions, a point $(x,y) \in \mathcal{F}$ is critical, if the Karush-Kuhn-Tucker (KKT) conditions are satisfied. Let $L(x,y,\lambda,\mu,\gamma)$ the Lagrangian function associated to Problem (\ref{eq:problem}) then we can write KKT conditions.

\begin{proposition}[Optimality conditions (Necessary)]\label{prop:KKT}

Let $(x^*,y^*) \in \R^{n+m}$, with $(x^*,y^*) \in \mathcal{F}$, a local optimum for Problem (\ref{eq:problem}). Then there exist three multipliers $\lambda^* \in \R^n$, $\mu^* \in \R^n, \gamma^* \in \R$ such that:
\begin{equation}
 \begin{aligned}
  &\nabla_x L(x^*,y^*\lambda^*,\mu^*,\gamma^*)= \nabla_x f(x^*,y^*)+\lambda^*-\mu^*+\gamma^*=0\\
 &\nabla_y L(x^*,y^*,\lambda^*,\mu^*,\gamma^*)=\nabla_y f(x^*,y^*) =0 \\
    &\lambda^*_i(l_i-x_i^*)=0,\ \forall i\\
 &\mu^*_i(x_i^*-u_i)=0,\ \forall i\\
   & \lambda^*,\mu^*\ge0 \\
 \end{aligned}
\end{equation}
\end{proposition}

From the first condition we have:
\begin{equation}
 \nabla_x f(x^*,y^*)-\lambda^*+\mu^*+\gamma^*=0
\end{equation}

Then there are three possible cases:
\begin{equation}
 \frac{\partial f(x^*,y^*)}{dx_i} = \begin{cases} -\mu_i^* -\gamma^* \hspace{1cm} x^*_i =u \\
 -\gamma^*+\lambda^*_i \hspace{1cm} x^*_i =l \\
 -\gamma^* \hspace{1.65cm} l<x^*_i <u 
\end{cases}
\end{equation}
Then if $x^*_i>l_i$: 
\begin{equation}
 \frac{\partial f(x^*,y^*)}{dx_i} \le \frac{\partial f(x^*,y^*)}{dx_h}, \forall h
\end{equation}

After writing KKT optimality condition we focus on feasible direction in a feasibile point.

A vector $d\in \R^{n+m}$ is partitioned as follows
$$
d=\left(
\begin{array}{c}
d_x\\
d_y
\end{array}
\right ),
$$
where $d_x\in R^n$ and $d_y\in R^m$.

Given $(x,y) \in \mathcal{F}$, the set of feasible directions in $(x,y)$ is the cone
\begin{equation}
 \mathcal{D}(x,y)=\{ d \in \R^{n+m}: \mathds{1}^Td_x=0, d_i\ge 0 \ \forall i \in L(x), d_i\le 0 \ \forall i \in U(x)\}
\end{equation}
where
\begin{equation}
 \begin{aligned}
  &L(x)=\{ i: \ x_i=l_i\}\\
  &U(x)=\{ i: \ x_i=u_i\}
 \end{aligned}
\end{equation}
Given $(\bar x,\bar y) \in \mathcal{F}$, we say that $(\bar x,\bar y)$ is a {\it critical point} if
$$
\nabla f(\bar x,\bar y)^Td\ge 0\quad\quad \forall d\in  \mathcal{D}(\bar x,\bar y).
$$
We can state the following result.
\begin{proposition}\label{optimality}
A point $(\bar x,\bar y) \in \mathcal{F}$ is a critical point if and only if
\begin{equation}\label{on_x}
\begin{aligned}
&\nabla_xf(\bar x,\bar y)^Td_x\ge 0 \quad \forall d_x\in R^n & \\ 
&\text{s.t.} \quad \mathds{1}^Td_x=0,\quad d_i\ge 0 \ \forall i \in L(\bar x), \quad d_i\le 0 \ \forall i \in U(\bar x)&
\end{aligned}
\end{equation}
\begin{equation}\label{on_y}
 \nabla_yf(\bar x,\bar y)=0.
\end{equation} 
\end{proposition}

In correspondence to a feasible point $(x,y)$ we introduce the index sets
$$
R(x)=L(x)\cup C(x)
$$
$$
S(x)=U(x)\cup C(x),
$$
where 
$$
C(x)=\{i: l_i<x_i<u_i\}.
$$
We can state the following propositions (Propositions 2.2 and 2.3 in Jota2009).
\begin{proposition}\label{2.2}
 A feasible point $(\bar x,\bar y)$ is a critical point if and only if for each pair $(i,j)$,
$i\in R(\bar x)$, $j\in S(\bar x)$, we have
\begin{equation}\label{on_RS}
 {{\partial f(\bar x,\bar y)}\over{\partial x_i}}\ge
 {{\partial f(\bar x,\bar y)}\over{\partial x_j}}
\end{equation}
\begin{equation}\label{on_y2}
\nabla_yf(\bar x,\bar y)=0.
\end{equation}
\end{proposition}
\begin{proposition}\label{2.3}
 Let $\{(x^k,y^k)\}$ be a sequence of feasible points convergent to a point $(\bar x,\bar y)$.
Then, for sufficiently large values of $k$, we have
$$
R(\bar x)\subseteq R(x^k) \quad \quad {\rm and}\quad \quad S(\bar x)\subseteq S(x^k).
$$
\end{proposition}

\subsection{Set of sparse feasible directions}
In our decomposition framework we will employ feasible directions having only two
nonzero components.
Then, in this subsection we introduce these sparse feasible directions and we show their important properties.

Given $i, j\in  \{1, \ldots ,n\}$, with $i\ne j$,
we indicate by $d^{i,j}\in  \R^{n+m}$ such that
\begin{equation}\label{eq:direction}
d_h^{i,j}= 
\begin{cases}
1, \quad \text{    } h=i\\
-1, \text{    } \text{    } h=j\\
0, \quad \text{    } \text{otherwise}
\end{cases}
\end{equation}
Note that, by definition, the components $d_h^{i,j}$ with $h=n+1,\ldots,n+m$ are always set to zero.

Given $(x, y) \in \mathcal{F}$ and the corresponding index sets $R(x)$ and $S(x)$, we indicate by $D_{RS}(x,y)$
the set of directions $d^{i,j}$ with $i \in R(x)$ and $j \in S(x)$, namely
$$
D_{RS}(x,y)=\cup_{i\in R(x),j\in S(x)}d^{i,j}.
$$
\begin{proposition}\label{3.1}
Let $(\bar x,\bar y)$ be a feasible point. For each pair $i \in R(x)$ and $j \in S(x)$, the
direction $d^{i,j}\in \R^{n+m}$ is a feasible direction at $(\bar x,\bar y)$, i.e. $d \in D(\bar x,\bar y)$.
\end{proposition}
\begin{proposition}\label{3.2}
A feasible point $(\bar x,\bar y)$
 is a critical point if and only
\begin{equation}\label{on_x2}
\nabla f(\bar x,\bar y)^Td^{i,j}\ge 0\quad\quad \forall d^{i,j}\in D_{RS}(\bar x,\bar y)
\end{equation}
\begin{equation}\label{on_y3}
 \nabla_y f(\bar x,\bar y)=0.
\end{equation} 
\end{proposition}
Given a feasible point $(\bar x,\bar y)$, a pair $i\in R(\bar x)$ and $j\in S(\bar x)$ such that
$$
\nabla f(\bar x,\bar y)^Td^{i,j}<0
$$
is said a {\it Violating Pair} (VP).

A violating pair $(i^\star,j^\star)$ such that 
\begin{equation}\label{mvp}
 \nabla f(\bar x,\bar y)^Td^{i^\star,j^\star}\le \nabla f(\bar x,\bar y)^Td^{i,j} \quad \forall i\in R(\bar x), \ j\in S(\bar x).
\end{equation}
is the {\it Most Violating Pair} (MVP).

\subsection{Armijo-Type Line Search Algorithm}
In this section, we briefly describe the well-known Armijo-type line search along a feasible descent direction. 
Let $d^{k} \in \mathcal{D}(x_k)$, $x^{k} \in \mathcal{F}$. We denote by $\Delta_{k}$ the maximum feasible step along $d^{k}$. It is easy to see that:
\begin{equation*}
\Delta_k= \min \{ x^k_{j(k)}-l_{j(k)}, u_{i(k)}-x^k_{i(k)}\}
\end{equation*}

\begin{algorithm}[ht]
 \KwData{Given $\alpha > 0$, $\delta \in (0,1)$, $\gamma \in (0, 1/2)$ and the initial stepsize $\Delta^{(k)} =\min \{ x^k_{j(k)}-l, u-x^k_{i(k)}\}$ }
 %\KwResult{A feasible step $\lambda$}
 Set $\alpha = \Delta^{(k)}$\\
 \While{$f(x^{k},y^k) + \alpha d^{k}) > f(x^{k},y^k) + \gamma \alpha \nabla_x f(x^{k},y^k)^T d^{k}$}{
  Set $\alpha = \delta \alpha$
 }
 \caption{Armijo-Type Line Search}
\end{algorithm}

Then at iteration $k+1$ we have:
\begin{equation*}
x^{k+1}_{j(k)}=\begin{cases}
 l_{j(k)} \ &\alpha_k=x^k_{j(k)}-l_{j(k)}\\
 x^k_{j(k)}-u_{i(k)} +x^k_{i(k)} \ &\alpha_k=u_{i(k)} -x^k_{i(k)}
 \end{cases}
\end{equation*}
and:
\begin{equation*}
x^{k+1}_{i(k)}=\begin{cases}
 x^k_{j(k)}-l_{j(k)}+x^k_{i(k)} \ &\alpha_k=x^k_{j(k)}-l_{j(k)}\\
 u_{i(k)} \ &\alpha_k=u_{i(k)}-x^k_{i(k)}
 \end{cases}
\end{equation*}
\subsection{Quadratic Line Search (QLS)}
In order to find a feasible step along a descent direction in $x^k$, we use a line search method called quadratic line search. The QLS algorithm procedure (Algorithm \ref{alg:qls}) starts from $\alpha_k = \Delta_k$ and decreases $\alpha_k$ until:
\begin{equation}
f(x_k+\alpha_kd_k) \le  f(x_k)- \gamma (\alpha_k||d_k||)^2
\end{equation}
 where $d_k$ is a descent direction in $x_k$.

Altough the most famous line search method is Armijo-Type, its most important drawback is that it can't guarantee that:
\begin{equation}
 \displaystyle \lim_{k\rightarrow \infty} ||x^{k+1}-x^{k}|| =0
\end{equation}

 \begin{algorithm}[ht]
 \KwData{Given $\alpha > 0$, $\delta \in (0,1)$, $\gamma \in (0, 1/2)$ and the initial stepsize $\Delta^{k} =\min \{ x^k_{j(k)}-l, u-x^k_{i(k)}\}$ }
 %\KwResult{A feasible step $\lambda$}
 Set $\alpha = \Delta^{k}$\\
 \While{$f(x^{k},y^k) + \alpha d^{k}) > f(x^{k},y^k) - \gamma \left(\alpha ||d^{k}||\right)^2$}{
  Set $\alpha = \delta \alpha$
 }
 \caption{QLS Line Search}\label{alg:qls}
\end{algorithm}

QLS algorithm has also the same convergency properties \cite{sciandrone-galligari-dilorenzo} of Armijo-type.
\section{The decomposition framework}
As already discussed, we partition the vector of variables into two blocks in order to take into account
the structure of the feasible set and possibly the form of the objective function (see, for instance,
the formulation of the risk parity problem, where the objective function is convex w.r.t. the scalar variable $\theta$).
The first block contains the constrained variables $x$, the second block contains
the unconstrained variables $y$. 

A first possibility can be that of defining a two-blocks Gauss-Seidel algorithm.
According to this scheme, at each iteration, the two component vectors $x$ and $y$ are
sequentially updated by performing  minimization steps (either exact or inexact) by  suitable descent techniques.
Globally convergent results of Gauss-Seidel algorithms (both exact and inexact) have been established in \cite{}, \cite{}, \cite{}.

We present here a block descent algorithm where a further level of decomposition is
introduced with respect to the block component $x$. 
More specifically, at each iteration, only two variables are updated, those corresponding
to a {\it Violating Pair}, by performing an inexact line search along a feasible and descent direction.
In order to guarantee convergence, the {\it Most Violating Pair} must be selected at least periodically, say every $M$ iterations.
We will discuss in the section of the computational experiments the role and the influence of the parameter $M$.

The adoption of a decomposition strategy with respect to the subvector $x$ is suitable
whenever the number $n$ of variables is large.

Note that the properties of the standard Armijo-type line search do not guarantee, without further assumptions
on the descent search direction $d^k$, that the distance between successive points tends to zero, which is a usual requirement of decomposition methods. 
This motivates the employment of the Quadratic Line Searck (QLS) defined in Algorithm \ref{alg:proximal} and based on the acceptance condition
$$
f((x^k,y^k)+\alpha^kd^k)\le f((x^k),y^k))-\gamma (\alpha^k)^2\|d^k\|^2.
$$
Concerning the unconstrained block component $y$, we do not specify the updating rule, but we state the following assumption that
could be satisfied, in practice, by different techniques depending on the hypothesis on $f$.
\par\medskip\noindent
{\bf Assumption on the updating rule of} $y$.
\par\medskip\noindent
\begin{itemize}
\item[(i)] For each $k$ we have $f(x^k,y^{k+1})\le f(x^k,y^k)$
\item [(ii)] If
$$
\lim_{k\to\infty} \left(f(x^k,y^{k+1})- f(x^k,y^k)\right)=0
$$
then
$$
\lim_{k\to\infty}\|y^{k+1}-y^k\|=0
$$
and
$$
\lim_{k\to\infty}\nabla_y f(x^k,y^k)=0.
$$
\end{itemize}


%At every step $k$ we choose a random subset $W^k\subset \{1,..,n\}$ such as
%\begin{equation}\label{eq:lambda}
%\frac{|W^k|}{n} \times 100 = \lambda
%\end{equation}
%For each $w \in W^k$, we compute the partial derivative $\frac{\partial f(x,\theta)}{\partial x_w}$ and 
%we select the MVP among the indexes in $W^k$. If we don't find a violating pair in $W^k$, 
%we randomly add indexes until we find one, and we use this violating pair to build the descent direction. To assure the global convergence properties, we evaluate the MVP among the full gradient $\nabla_x f(x,\theta)$ every $M$ iterations. \\
The algorithm is formally described below.

\begin{algorithm}[ht]
 \KwData{Given the initial feasible point $(x^{0}, y^{0})$}
 Set $k = 0$\\
 \While{(not convergence)}{
  Compute $y^{k+1}$ such that Assumptions (i) and (ii) hold\\
 \eIf{$k \enskip \text{mod} \enskip M = 0$}
  {
  Let $(i(k), j(k))$ be a Most Violating Pair\\ 
  }
  {
  Let $(i(k), j(k))$ be a Violating Pair\ 
  }
  Compute a step $\alpha^{k}$  along the direction $d^k=d^{i(k),j(k)}$ by QLS\\
  Set $x_{i(k)}^{k+1} = x_{i(k)}^{k} + \alpha^{k}$, $x_{j(k)}^{k+1} = x_{j(k)}^{k} - \alpha^{k}$  \\

  Set $k = k + 1$
 }
 \caption{Decomposition Algorithm}
 \label{alg:decMVP}
\end{algorithm}
\par\bigskip\noindent
We can prove the following global convergence result.
\begin{proposition}
Suppose that the level set $\mathcal{L}_0$ is a compact set. Let $\{(x^k, y^k)\}$ be the sequence of points generated by the decomposition algorithm. Then
$\{(x^k, y^k)\}$ admits limit points and each limit point is critical for Problem (\ref{eq:problem}).
%\end{itemize}
\end{proposition}

\begin{proof}
The instructions of Algorithm \ref{alg:decMVP} imply
$$
f(x^{k+1}, y^{k+1})\le f(x^{k}, y^{k+1}) \leq f(x^{k}, y^{k}),
$$
so that, the points of the sequence $\{(x^{k}, y^{k})\}$ belongs to the compact set $\mathcal{L}_0$.
We also have
\begin{equation}\label{on_funct}
 \lim_{k\to\infty} f(x^k,y^{k+1})=\lim_{k\to\infty} f(x^{k},y^{k})=\bar f>-\infty
\end{equation}
Let $(\overline{x},\overline{y})$ be a limit point of $\{(x^k, y^k)\}$, i.e. there exists an infinite subset $K \subseteq N$ such that
\begin{equation}\label{eq:asim}
\lim_{k \in K, k \rightarrow \infty} (x^k, y^k) = (\overline{x},\overline{y})
\end{equation}
By contradiction, let us assume that $(\overline{x},\overline{y})$ is not a critical point. 
In this case, at least one of the following conditions holds:
\begin{subequations}
\begin{align}
&\nabla_y f(\overline{x},\overline{y}) \neq 0  \label{eq:aa}\\
&\exists \enskip i, j \enskip  \text{s.t.} \enskip d^{i,j}\in D(\bar x) \enskip  \text{and} \enskip  \nabla_x f(\overline{x},\overline{y})^T d^{i,j} = -\eta < 0 \label{eq:bb}
\end{align}
\end{subequations}
Suppose that (\ref{eq:aa}) holds.
From (\ref{on_funct}), Assumption (ii) on the updating rule of $y^k$, and the continuity of the gradient we get
$$
\lim_{k\to\infty}\nabla_y f(x^k,y^k)=\nabla_y f(\overline{x},\overline{y})=0,
$$
and this contradicts (\ref{eq:aa}).

Now assume that (\ref{eq:bb}) holds.
For each $k$, a stepsize $\alpha^k>0$ is computed by QLS along the descent direction $d^{i(k),j(k)}$.
Then we can write
\begin{equation}\label{red_funct}
 f(x^{k+1},y^{k+1})\le f(x^k,y^{k+1})-\gamma (\alpha^k)^2\|d^{i(k),j(k)}\|^2=f(x^k,y^{k+1})-\gamma\|x^{k+1}-x^k\|^2,
\end{equation}
from which, recalling (\ref{on_funct}) and that $\|d^{i(k),j(k)}\|^2=2$, we obtain
\begin{equation}\label{dst_x}
 \lim_{k\to\infty}\|x^{k+1}-x^k\|=\lim_{k\to\infty}\alpha^k=0.
\end{equation} 
From (\ref{on_funct}) and Assumption (ii) on the updating rule of $y^k$ we also have
\begin{equation}\label{dst_y}
 \lim_{k\to\infty}\|y^{k+1}-y^k\|=0.
\end{equation}
For each $k\in K$, let $v(k)$ be the integer such that $k+v(k)$ is an iteration
where the Most Violating Pair is selected. Note that we have
$$
0\le v(k) \le M.
$$
From (\ref{dst_x}) and (\ref{dst_y}) we obtain
\begin{equation}\label{cnv_x}
 \lim_{k\in K,k\to\infty}x^{k+v(k)}=\bar x
\end{equation}
\begin{equation}\label{cnv_y}
 \lim_{k\in K,k\to\infty}y^{k+v(k)+1}=\bar y.
\end{equation}
We introduce the index set 
$$
K_1=\{h: \ h=k+v(k), \ k\in K\}.
$$
By definition, for all $k\in K_1$ the Most Violating Pair $(i(k),j(k))$ is selected. Furthermore, we have
\begin{equation}\label{cnv2_x}
 \lim_{k\in K_1,k\to\infty}x^{k}=\bar x
\end{equation}
\begin{equation}\label{cnv2_y}
 \lim_{k\in K_1,k\to\infty}y^{k+1}=\bar y.
\end{equation}
Since $i(k)$ and $j(k)$ belong to the finite set $\{1,\ldots ,n\}$, we can extract a further subset (that we relabel by $K_1$) such that
$$
i(k)=i^\star \quad\quad j(k)=j^\star\quad\quad \forall k\in K_1.
$$
Note that $d^{i,j}\in D(\bar x,\bar y)$, so that, recalling Proposition \ref{2.3}, we obtain for
$k\in K_1$ and $k$ sufficiently large
\begin{equation}\label{dij_dk}
 d^{i,j}\in D(x^k,y^k).
\end{equation}
For all $k\in K_1$, as $(i^\star,j^\star)$ is the Most Violating Pair, using (\ref{dij_dk}) we can write
\begin{equation}\label{mvp_ij}
 \nabla_xf(x^k,y^{k+1})^Td^{i^\star,j^\star}\le \nabla_xf(x^k,y^{k+1})^Td^{i,j}<0.
\end{equation}
Then, $d^{i^\star,j^\star}$ is a feasible and descent direction, and hence a stepsize $\alpha^k$ is computed along it by QLS.
Further, from (\ref{dst_y}) and (\ref{eq:bb}) it follows
\begin{equation}\label{desc_mvp}
 \nabla_xf(\bar x,\bar y)^Td^{i^\star,j^\star}<0.
\end{equation}
Now let us distinguish two cases:
\par\medskip\noindent
Case (I). There exists an integer $\ell$  such that
\begin{equation}\label{caseI}
 \alpha^{k+\ell(k)}<\Delta^{k+\ell(k)}
\end{equation}
for all $k\in K_1$ and for some $\ell(k)\le \ell$.
\par\medskip\noindent 
Case (II). For all $k\in K_1$ and $m=0,\ldots ,2n$ we have
\begin{equation}\label{caseII}
 \alpha^{k+m}=\Delta^{k+m}.
\end{equation}
{\it Case} (I)
\par\medskip\noindent
Without loss of generality assume $\ell (k)=0$.
Otherwise we can reason on the subsequence $\{x^k\}_{k+l(k)}$ that, by definition,
is a subsequence where the  MVP is selected again (see the new instruction) since $\alpha^{k+l(k)-s}=\Delta^{k+l(k)-s}$,
for $s=0,1,\ldots ,l(k)$,
and $\alpha^{k+l(k)}<\Delta^{k+l(k)})$.

The instructions of QLS imply for all $k\in K_1$
\begin{equation}\label{eq:arm1}
f(x^k + \frac{\alpha^k}{\delta} d^{i^\star,j^\star}, y^{k+1}) - f(x^k,y^{k+1}) > -2\gamma \left(\frac{\alpha^k}{\delta}\right)^2 
\end{equation}
Using the Mean Value Theorem, we can write
\begin{equation}\label{eq:arm2}
f(x^k + \frac{\alpha^k}{\delta} d^{i^\star,j^\star}, y^{k+1}) = f(x^k, y^{k+1}) + \frac{\alpha^k}{\delta} \nabla_x f(z^k, y^{k+1})^T d^{i^\star,j^\star}
\end{equation}
where $z^k = x^k + \vartheta_k \frac{\alpha^k}{\delta} d^{i^\star,j^\star}$ and $\vartheta_k \in (0,1)$. 
From (\ref{eq:arm2}) and (\ref{eq:arm1}), for $k\in K_2$ and $k$ sufficiently large we have
\begin{equation}\label{eq:nabla}
\nabla_x f(z^k, y^{k+1})^T d^{i^\star,j^\star}  >- 2\gamma \frac{\alpha^k}{\delta} 
\end{equation}
Using (\ref{dst_x}) and (\ref{dst_y}) and we obtain
$$
\lim_{k \in K_1, k \rightarrow \infty} z^k = \lim_{k \in K_1, k \rightarrow \infty} x^k + \vartheta_k \frac{\alpha^k}{\delta} d^{i^\star,j^\star} = \overline{x}
$$
$$
\lim_{k\in K_1,k\to\infty}y^{k+1}=\bar y.
$$
Taking the limits in (\ref{eq:nabla}) for $k\in K_1$ and $k\to\infty$ we have
\begin{equation}\label{ddirmg0}
\nabla_x f(\overline{x},\overline{y})^T d^{i^\star,j^\star} \geq 0,
\end{equation}
which contradicts (\ref{desc_mvp}).
\par\bigskip\noindent
{\it Case} (II).
\par\medskip\noindent
For all $m=0,\ldots ,2n$ we have that at least one of these two possible cases hold
 \begin{subequations}
\begin{align}
 i(k+m)&\in R(x(k+m))\quad \quad i(k+m)\notin R(x(k+m+1))\label{eq:SetCasesA}\\
 j(k+m)&\in S(x(k+m))\quad \quad j(k+m)\notin S(x(k+m+1))\label{eq:SetCasesB}
\end{align}
\end{subequations}
Now we define two sets $\Gamma_1,\Gamma_2$ in $\{x^{k}\}_{k \in K}$ such that the first one contains all indexes $m \in\{0,\ldots,2n\}$ such that 
(\ref{eq:SetCasesA}) holds.
The second one contains all indexes $m \in\{0,\ldots,2n\}$ such that (\ref{eq:SetCasesB}) holds.

Since $|\Gamma_1|+|\Gamma_2|\ge 2n+1$, one of these sets contains a number of elements greater than $n$. 
Without loss of generality assume $|\Gamma_1|> n$.
We can say that there exist $ \hat i \in \{1,\ldots,n\}$ and $l(k),m(k)$ such that:
 \begin{equation}
  k\le l(k) <m(k)\le 2 n
 \end{equation}
and
\begin{equation}
 i(k) = i(l(k))=i(m(k))=\hat i.
\end{equation}
We can define a subset $K_1 \subseteq K$ such that $\forall k_i \in K_1$ we have
\begin{equation}
 i(k_i)=\hat i
\end{equation}
and
\begin{equation}
 k_i <k_{i+1} \le k_i+2n
\end{equation}
From the MVP rule it follows
\begin{equation}
 \frac{\partial f(x^{k_i},y^{k_i+1})}{\partial x_{\hat i}} \le \frac{\partial f(x^{k_i},y^{k_i+1})}{\partial x_{h}}, \ \forall h \in R(x^{k_i})
\end{equation}
For all $k_i \in K_1$, there exists $p(k_i)$, with  $k_i <p(k_i)<k_{i+1}$, such that
\begin{equation}
 \hat i \not \in R(x^{p(k_i)})\quad\quad \hat i \in \in R(x^{p(k_i)+1}).
\end{equation}
Then, again from the MVP rule, we must have
\begin{equation}
 \frac{\partial f(x^{p(k_i)},y^{p(k_i)+1})}{\partial x_{\hat i}} \ge \frac{\partial f(x^{p(k_i)},y^{p(k_i)+1})}{\partial x_{h}}, \ \forall h \in S(x^{p(k_i)})
\end{equation}
From (\ref{dst_x}) and (\ref{dst_y}), recalling that  $p(k_i)-k_i \le 2n$, we have
$$
 \lim_{k_i\rightarrow \infty} x^{p(k_i)}=\overline{x}
$$
$$
 \lim_{k_i\rightarrow \infty} y^{p(k_i)+1}=\overline{y}
$$
Note that $d^{i,j}\in D(\bar x,\bar y)$, so that, recalling Proposition \ref{2.3}, we obtain for
$k\in K_1$ and $k$ sufficiently large
\begin{equation}\label{dij_dka}
 d^{i,j}\in D(x^{k_i},y^{k_i+1}),
\end{equation}
\begin{equation}\label{dij_dkbis}
 d^{i,j}\in D(x^{p(k_i)},y^{p(k_i)+1}),
\end{equation}
i.e.,
$$
i\in R(x^{k_i})\quad\quad j\in S(x^{k_i})
$$ 
$$
i\in R(x^{p(k_i)})\quad\quad j\in S(x^{p(k_i)}).
$$
Then  we can write
\begin{equation}\label{eq:direction1}
\begin{aligned}
 \frac{\partial f(x^{k_i},y^{k_i+1})}{\partial x_{\hat i}} &\le \frac{\partial f(x^{k_i},y^{k_i+1})}{\partial x_{i}}\\
 \frac{\partial f(x^{p(k_i)},y^{p(k_i)+1})}{\partial x_{\hat i}} &\ge \frac{\partial f(x^{p(k_i)},y^{p(k_i)+1})}{\partial x_{j}}
 \end{aligned}
\end{equation}
Taking the limits for $k_i\in K_1$ and $k_i\to\infty$ and recalling the continuity of the gradient we obtain
$$
 \frac{\partial f(\overline{x},\overline{y})}{\partial x_i} - \frac{\partial f(\overline{x},\overline{y})}{\partial x_{j}} \ge 0,
$$
which contradicts (\ref{eq:bb}).
\end{proof}

\section{Application to Risk Parity}
Let us consider the problem
\begin{equation}\label{eq:rpfordec}
\begin{aligned}
& \underset{x, \theta}{\text{min}}
&&f(x,\theta)=\sum_{i=1}^n \left(x_i (\Sigma x)_i - \theta\right)^2\\
& \text{s.t.}
&& l \leq x \leq u\\
&&& \mathds{1}^T x = 1 \\
\end{aligned}
\end{equation}
Problem (\ref{eq:rpfordec}) is an ERC problem and it's easy to see that it is a particular case of Problem (\ref{eq:problem}). Thus we can use the decomposition algorithm just exposed to solve it.\\
$f(x,\theta)$ is clearly non-convex, but it is quadratic and strictly convex with respect to $\theta$.

\subsection{Proximal Point modification}
In this subsection we propose a \emph{Risk-Parity} version of general decomposition algorithm described above, using proximal point approximation.

Let us redefine selected variable $x_{i(k)},x_{j(k)}$ as $x_{i},x_{j}$ for ease of notation, and let us consider the step when $\theta$ is fixed (i.e $f(x,\theta)=f(x)$).

The idea behind proximal point modification is that at every iteration $k$, considering $\theta$ fixed, it's easy to solve the subproblem:
\begin{align}
 &\min_{x_i,x_j}h(x_i,x_j)= f(x_i,x_j)+ \frac{1}{2} \tau||(x_i,x_j)-(x_i^k,x_j^k)||^2\\
 &x_i+x_j = \underbrace{1-\sum_{h \ne i,j} x^k_h}_{c^k}\\
 &l \le x_i,x_j\le u
 \end{align}
where $f$ is defined in (\ref{eq:rpfordec}) and $\tau>0$.

In fact, due to simplex constraint, the objective function depends only on one of the selected variables (e.g. $x_j$) and $f$ become 4-degree polynomial in $x_j$.

Because $h$ is continuously differentiable, it admits minimum in a compact set and we have to search it between zeros of $h'(\xi)$ and the limit points of the feasible set.

Hence, at every iteration $k$, we can define the set of possible global minima as:
\begin{equation}
 O_k = \{ \xi < x_j^k: h'(\xi)=0\} \cup \{\min\{l_j,x_i^k\} \}
\end{equation}

then we set:
\begin{equation}
x_j^{k+1}= \arg \min_{\xi \in O_k} \{h(\xi)\}
\end{equation}
and:
\begin{equation}
x_i^{k+1}= c^{k}-x^{k+1}_j
\end{equation}



\begin{algorithm}[ht]
 \KwData{Given the initial feasible point $(x^{0}, \theta^{0})$}
 Set $k = 0$\\
 \While{(not convergence)}{
   Compute $\theta^{k+1}$ such that $f(x^{k},\theta^{k+1})\le f(x^{k},\theta^k)$ and $\nabla_\theta f(x^{k},\theta^{k+1})=0$\\
 \eIf{$k \enskip \text{mod} \enskip M = 0$}
  {
  Let $(i(k), j(k))$ be the MVP\\ 
  }
  {
  Let $(i(k), j(k))$ be a violating pair\\ 
  }
  Compute $\displaystyle x_{i(k)},x_{j(k)}\in \arg \min_{\xi,\zeta} f(\xi,\zeta)+\frac{1}{2}\tau ||(\xi,\zeta)-(x_{i(k)}^k,x_{j(k)}^k)||^2$\\
  Set $k = k + 1$
 }
 \caption{Decomposition Algorithm with proximal point}
 \label{alg:proximal}
\end{algorithm}
We can prove the following global convergence results.
\begin{proposition}
Suppose that the level set $\mathcal{L}_0$ is a compact set. Let $\{(x^k, \theta^k)\}$ be the sequence of points generated by the decomposition algorithm with proximal point modification. Then
\begin{itemize}
\item $(x^k, \theta^k) \in \mathcal{L}_0, \enskip \forall k$ 
\item  $\{(x^k, \theta^k)\}$ admits limit points and each limit point is critical for Problem (\ref{eq:rpfordec})
\end{itemize}
\end{proposition}
\begin{proof}
The firts assertion is given by algorithm statement which say:
\begin{equation}
f(x^{k+1},\theta^{k+1})\le f(x^{k+1},y^k)\le f(x^k,\theta^k)- \underbrace{\frac{1}{2}\tau ||x^{k+1}-x^{k}||^2}_{>0}
\end{equation}
then the sequence $\{(x^k,\theta^k)\} \in \mathcal{L}_0$.

To prove the second assertion it is necessary to show that
\begin{equation}
\lim_{k \rightarrow \infty} ||(x^{k+1},\theta^{k+1})-(x^{k},y^{k})||=0
\end{equation}
then the convergence can be completed as in previous proposition.

Let us start to prove that $||x^{k+1}-x^{k}||\to0$.

From the $\mathcal{L}_0$ compactness we can say that the sequence $\{(x^k,\theta^k)\}$ has a subsequence $K \subset \{0,1,\ldots\}$ convergent to $(\overline{x},\overline{y}) \in \mathcal{L}_0$. Then for the continuity of $f$ we have:
\begin{equation}
\lim_{k \in K, k \rightarrow \infty}f(x^k,\theta^k)=f(\overline{x},\overline{\theta})= \overline{f}
\end{equation}

The instruction of algorithm imply that:
\begin{equation}
f(x^{k+1},\theta^{k+1})\le f(x^{k+1},\theta^{k})\le f(x^k,\theta^k)
\end{equation}
then the sequence $f(x^k,\theta^k)$ is decreasing and lower-bounded by $\overline{f}$ then we have:
\begin{equation}
\lim_{k \in K, k \rightarrow \infty} f(x^{k+1},\theta^{k})-f(x^{k},\theta^{k})=0
\end{equation}

Proximal point step imply that:
\begin{equation}
f(x^{k+1},y^{k})+\frac{1}{2}\tau||x^{k+1}-x^{k}||^2 \le f(x^{k},y^{k})
\end{equation}

Taking the limit for $k \in K, k \to \infty$ we obtain:
\begin{equation}
\lim_{k \in K,k \to \infty} ||x^{k+1}-x^{k}||=0
\end{equation}

Now we have to prove that $|\theta^{k+1}-\theta^{k}|\to 0$, using the property that $f$ is quadratic and strictly convex with respect to $\theta$. 
We have to prove the following lemma:
\begin{lemma}
Let $f(x,\theta)$ defined in (\ref{eq:rpfordec}), then\\ if $\lim_{k \to \infty} f(x^{k},\theta^k)-f(x^{k},\theta^{k+1})=0$ then:
\begin{equation}
 \lim_{k\to \infty} |\theta^{k+1} -\theta^k| = 0 
\end{equation}
\end{lemma}

\begin{proof}
The instructions of the algorithm imply that at every iteration $k$, we have to select the optimal step $\alpha_k^*$ along $-\nabla_{\theta}f(x^{k},\theta^{k})$.

Because of $f$ is strictly convex quadratic function with respect to $\theta$ we have:
\begin{equation}\label{eq:quadratic}
f(x^{k},\theta^k-\alpha^*_k\nabla_{\theta}f(x^{k},\theta^{k})) = f(x^{k},\theta^{k})-\alpha^*_k |\nabla_{\theta}f(x^{k},\theta^{k})|^2
\end{equation}
with:
\begin{equation}\label{eq:upTheta}
 \theta^{k+1}= \theta^{k}-\alpha^*_k\nabla_{\theta}f(x^{k},\theta^{k})
\end{equation}

From hypothesis we have that:
\begin{equation}
 \lim_{k \to \infty} f(x^{k+1},\theta^k)-f(x^{k+1},\theta^{k+1})=0
 \end{equation}

and from (\ref{eq:upTheta}) and (\ref{eq:quadratic}):
\begin{equation}
  \lim_{k\to \infty} |\theta^{k+1} -\theta^k| = 0 
\end{equation}
\end{proof}
 
Next convergence steps are equal to previous proposition.
\end{proof}

\begin{oss}
In computational experiments we use $\tau = 0$ and compute an exact line search along direction $d^{i(k),j(k)}$. In next section we will show that in practical cases exact line search performs very well especially increasing number of variables.
\end{oss}

\subsection{Implementation guidelines}
First of all, we can write the derivative of the objective function of Problem (\ref{eq:rpfordec}) with respect to $x$ and $\theta$ as:
\begin{equation}\label{eq:gradxdec}
\frac{\partial f(x,\theta)}{\partial x_j} = 2 \sum_{i=1}^n \left(x_i (\Sigma x)_i - \theta\right)\Sigma_{ij}x_i + 2(x_j (\Sigma x)_j - \theta)(\Sigma x)_j
\end{equation}
and
\begin{equation}\label{eq:gradthetadec}
\nabla_{\theta} f(x,\theta)= -2\sum_{i=1}^n \left(x_i (\Sigma x)_i - \theta\right)
\end{equation}
\subsubsection{Matrix product}
As we can see from (\ref{eq:gradxdec}) and (\ref{eq:gradthetadec}), at every iteration $k$ we must compute:
\begin{equation}
\Sigma  x^{k}
\end{equation}
This scalar product is highly time consuming and we therefore use another approach to compute it. At every iteration, only 2 variables changes value, more precisely:
\begin{equation}
x_{i(k)}^{k+1} = x_{i(k)}^{k} + \alpha^{k}
\end{equation}
\begin{equation}
x_{j(k)}^{k+1} = x_{j(k)}^{k} - \alpha^{k}
\end{equation}
Thanks to this updating scheme, we can write:
\begin{equation}
\Sigma x^{k+1} = \Sigma  x^{k} + \left[\Sigma_{i(k)} - \Sigma_{j(k)}\right]\alpha^{k} 
\end{equation}
where $\Sigma_{i}$ is the $i$-th column of $\Sigma$. This shrewdness dramatically reduces the computation time of the algorithm, especially for large values of $n$.

\subsubsection{Gradient computation}
We can split the gradient (\ref{eq:gradxdec}) computation in two parts. The first is:
\begin{equation}\label{eq:first1}
2 \sum_{i=1}^n \left(x_i (\Sigma x)_i - \theta\right)\Sigma_{ij}x_i
\end{equation}
while the second is:
\begin{equation}\label{eq:second}
2(x_j (\Sigma x)_j - \theta)(\Sigma x)_j
\end{equation}
Let us introduce the matrix $B \in \mathbb{R}^{n \times n}$ with elements:
\begin{equation}
B_{i,j} = \Sigma_{i,j}  x_i 
\end{equation}
and the vector $d \in \mathbb{R}^n$ with elements:
\begin{equation}
d_j = 2 (x_j (\Sigma x)_j - \theta) 
\end{equation}
We can now write the two terms of the gradient using $B$ and $d$; the first term (\ref{eq:first1}) becomes:
\begin{equation}
B d
\end{equation}
The second term (\ref{eq:second}) is simply:
\begin{equation}
d_j (\Sigma x)_j
\end{equation}
Finally, we have:
\begin{equation}\label{eq:newgradx}
\frac{\partial F(x,\theta)}{\partial x_j} =\left(B d\right)_j + d_j (\Sigma x)_j
\end{equation}
At every iteration $k$, we have:
\begin{equation}
B^{k}_{h} = \begin{cases}
		  		\Sigma_{h}x_h^{k} \quad \text{if } h = i(k) \vee h = j(k)\\\\
		  	    B^{k-1}_{h} \quad \text{     otherwise}
		  		\end{cases}
\end{equation}
where $B_{h}$ is the $h$-th column of $B$.

\subsubsection{Violating Pair selection}
In Algorithm \ref{alg:proximal} we choose a generic violating pair, without no assumptions on the \textquotedblleft quality " of this pair. In practice, we select a \textquotedblleft sufficient" violating pair instead, selecting the MVP among a subset of asset's indexes $\left\{1,.., n\right\}$. For each iteration $k$, let
\begin{equation}\label{eq:wk}
W^k \subset \left\{1,.., n\right\}
\end{equation}
a random subset of the asset's indexes such that
\begin{equation}\label{eq:wk2}
\frac{|W^k|}{n} \times 100 = \lambda
\end{equation}
Algorithm \ref{alg:modified} summarizes what discussed above.

\begin{algorithm}[ht]
 \KwData{Given the initial feasible point $(x^{0}, \theta^{0})$ and $\lambda$}
 Set $k = 0$\\
 \While{(not convergence)}{
   Compute $\theta^{k+1}$ such that $f(x^{k},\theta^{k+1})\le f(x^{k},\theta^k)$ and $\nabla_\theta f(x^{k},\theta^{k+1})=0$\\
 \eIf{$k \enskip \text{mod} \enskip M = 0$}
  {
  Let $(i(k), j(k))$ be the MVP\\ 
  }
  {
  Choose a random subset $W^k$ such that (\ref{eq:wk2}) holds\\
  Let $(i(k), j(k))$ be the MVP among indexes in $W^k$\\ 
  }
  Compute $\displaystyle x_{i(k)},x_{j(k)}\in \arg \min_{\xi,\zeta} f(\xi,\zeta)+\frac{1}{2}\tau ||(\xi,\zeta)-(x_{i(k)}^k,x_{j(k)}^k)||^2$\\
  Set $k = k + 1$
 }
 \caption{Decomposition Algorithm with $\lambda$}
 \label{alg:modified}
\end{algorithm}



\chapter{Computational Experiments}
\input{experiments}

\chapter{Conclusions}
\chapter{Acknowledgements}
\clearpage


\begin{thebibliography}{9}
\bibitem{markovitz}
  H. Marling, S. Emanuelsson,
  \emph{The Markowitz Portfolio Theory},
  2012.
\bibitem{sharpe}
  W. F. Sharpe,
  \emph{The Sharpe Ratio},
  1994.
\bibitem{diversification}
  Y. Choueifaty, T. Froidure, J. Reynier,
  \emph{Properties of the most diversified portfolio},
  2013. 
\bibitem{libro}
  G. Cornuejols, R. Tutuncu,
  \emph{Optimization methods in finance}, pp.138-140,
   2007.
\bibitem{tutuncu}
  X. Bai, K. Scheinberg, R. Tutuncu,
  \emph{Least-square approach to risk parity in portfolio selection},
  2013.   
\bibitem{bruder}
  B. Bruder, T. Roncalli,
  \emph{Managing Risk Exposures using the Risk Budgeting Approach},
  2012.   
\bibitem{roncalli}
  T. Roncalli,
  \emph{From portfolio optimization to risk parity},
  2012.
\bibitem{intr}
  T.Roncalli,
  \emph{Introducing Expected Returns into Risk Parity Portfolios: A New Framework for Asset Allocation},
  2014.   
\bibitem{erc}
  S. Maillard, T. Roncalli, J. Teiletche,
  \emph{On the properties of equally-weighted risk contributions portfolios},
  2009.
\bibitem{colucci}
  F. Cesarone, S. Colucci,
  \emph{Minimum Risk vs. Capital and Risk Diversification strategies for portfolio construction},
  2015.  
\bibitem{sdp}
  M. Haugh, G. Iyengar, I. Song, 
  \emph{A Generalized Risk Budgeting Approach to Portfolio Construction},
  2015.
\bibitem{newton}  
  B. Chaves, C. Hsu, F. Li, O.Shakernia,
  \emph{Efficient Algorithms for Computing Risk Parity Portfolio Weights},
  2012.
 \bibitem{nesterov}
  Y. Nesterov,
  \emph{Introductory Lectures on Convex Programming}, Chap. 4,
  1998.
\bibitem{spinu}
  F. Spinu,
  \emph{Newton's method for the risk parity problem},
  2013.
\bibitem{ccd}
  T. Griveau-Billion, J. Richard, T. Roncalli,
  \emph{A Fast Algorithm for Computing High-dimensional Risk Parity Portfolios},
  2013.
\bibitem{scrip}
  Y. Feng, D. P. Palomar,
  \emph{SCRIP: Successive Convex Optimization Methods for Risk Parity Portfolio Design},
  2015.
\bibitem{genetic}
   R. Hochreiter,
  \emph{An Evolutionary Optimization Approach to Risk Parity Portfolio Selection},
  2015.
  
\bibitem{erb}
  F. Cesarone, F. Tardella,		
  \emph{Equal Risk Bounding is better than Risk Parity for portfolio selection},
  2015.
  
\bibitem{snopt}
	P. E. Gill, W. Murray, M. A. Sanders,
	\emph{SNOPT: An SQP Algorithm for large-scale constrained optimization}, 
	2005.
  
\end{thebibliography}
\end{document}